\documentclass{book}
\usepackage{titling}
\usepackage{multirow}
\usepackage{setspace}
\usepackage{makeidx}
\usepackage{tikz}
\usepackage{subfigure}
\usepackage{braket}
\usepackage{graphicx}
\usepackage{array}

\usepackage{qcircuit}
\usepackage{graphicx}

\usepackage{tabularray}
\usepackage{amsmath}
\usepackage{circuitikz}
\usepackage{amssymb}
\usepackage{setspace}
\usepackage{amsmath}
\usepackage{xepersian}
\settextfont{XB Zar}
\linespread{1.5}



\makeatletter
\def\TikzBipolePath#1#2{\pgf@circ@bipole@path{#1}{#2}}
\def\CircDirection{\pgf@circ@direction}
\makeatother

% TikZ libraries `calc` needed now to tweak bracket.
\usetikzlibrary{backgrounds,fit,decorations.pathreplacing,calc}

\begin{document}
\tableofcontents
\newpage
\section*{مقدمه}	
در عصر حاضر بواسطه‌ی رشد و توسعه‌ی نظریه‌ی اطلاعات کوانتومی و سرمایه‌گذاری های مالی و انسانی بسیار در این زمینه،‌شاهد افزایش تعداد علاقمندان به این حوزه هستیم. در این پا
\newpage
\chapter{آشنایی با مفاهیم اولیه}
\section{کیوبیت}

یک کیوبیت\footnote{\lr{Qubit}}، معادل یک واحد اطلاعات کوانتومی می‌باشد. این مفهوم معادل مفهوم کلاسیک بیت\footnote{\lr{Binary Bit}} می‌باشد. به طور کلی هر کیوبیت حاوی دو بیت اطلاعات است. برای تبیین یک کیوبیت از خصوصیات سامانه های کوانتومی، بهره‌می‌بریم. کیوبیت یک سیستم کوانتومی با فضای دوبعدی است. برای تعیین این دوبعد می‌توان ا ز یکی از خصویات سامانه های کوانتومی استفاده کرد. 
\subsection{کیوبیت‌ واحد}
برخلاف بیت ها که مقادیر ثابت 0 یا 1 را به خود می‌گیرند؛ یک کیوبیت می‌تواند در یک حالت «برهمنهی کوانتومی\footnote{\lr{Quantum Superposition}}» باشد؛ این بدان معناست که یک کیوبیت بواسطه‌ی مشاهده ناظر به یکی از حالات 0 یا یک تبدیل شود. این مهم‌ترین مزیت استفاده از کیوبیت‌هاست. 

\pagebreak

بیان ریاضی یک کیوبیت ،در حالت برهمنهی، به شرح زیر است:
\vspace{0.5cm}
$$
\left\{
\begin{array}{ll}
	  \vert \psi \rangle = \alpha\vert 0 \rangle + \beta\vert 1 \rangle \\
	  \alpha^2 + \beta^2 = 1
\end{array}
\right.
$$

کت‌های $\vert 0 \rangle$ و $\vert 1 \rangle$ بیانگر پایه‌های فضای محاسباتی\footnote{\lr{Computational Basis Vectors}} هستند؛ و مقادیر $\alpha^2$ و $\beta^2$ بیانگر احتمال وقوع هر یک از این حالات، در صورت مشاهده، می‌باشند. نمایش بردار $\psi$ به شکل زیر است:
\begin{center}
\begin{tikzpicture}
	
	% Define the axes
	\draw[->] (-2,0)--(2,0) node[right]{$\vert 0 \rangle$};
	\draw[->] (0,-2)--(0,2) node[above]{$\vert 1 \rangle$};
	
	% Draw the vector
	\draw[black,-stealth] (0,0)--(1,2) node[anchor=south west]{$\vec{\psi}$};
	
	% Label the components of the vector
	\node[below] at (1,0) {$$};
	\node[left] at (0,2) {$$};
	
\end{tikzpicture}
\end{center}

در بسیاری از مواقع برای سهولت در محاسبات، عملگر‌ها و حالات کوانتومی به کمک ماتریس‌ها نمایش داده ‌می‌شوند. فرم ماتریسی هر یک از حالات ذکر شده در بالا به شرح زیر است:

\begin{equation}
	\vert 1 \rangle = \begin{pmatrix} 0 \\ 1 \end{pmatrix}
	\hspace{2cm}
	\vert 0 \rangle = \begin{pmatrix} 1 \\ 0 \end{pmatrix}	
\end{equation}

برای تعریف کیوبیت ها، راه های زیادی وجود دارد، حالات قطبش فوتون،‌اسپین الکترون،‌یا سطوح انرژی اتم،‌هریک می‌توانند تعیین کننده‌ی بردار‌های فضای کیوبیت باشند.

 به طور کلی، حالت کیوبیت یک بردار واحد در فضای برداری دو بعدی پیچیده است.\pagebreak
 
در بیشتر مدل های انتزاعی ما از جهان، یک ارتباط مستقیم بین عناصر انتزاع و دنیای واقعی وجود دارد، درست همانطور که طرح های یک معمار برای یک ساختمان با ساختمان نهایی مطابقت دارد. فقدان این ارتباط مستقیم در مکانیک کوانتوم باعث می شود که درک رفتار سیستم های کوانتومی دشوار باشد؛ با این حال، یک ارتباط غیرمستقیم وجود دارد، زیرا می توان حالت های کیوبیت را دستکاری و تبدیل کرد به وضعیت‌هایی که منجر به نتایج اندازه گیری می شود. نتایج حاصل از اندازه‌گیری به خواص مختلف حالت بستگی دارد. بنابراین، این حالت های کوانتومی دارای پیامدهای واقعی و قابل آزمایش تجربی هستند.

مفهوم کیوبیت، با «فهم رایج» ما از جهان فیزیکی اطراف ما مغایرت دارد. یک بیت کلاسیک مانند سکه است: یا رو یا پشت. برای سکه های غیراایده آل، ممکن است حالت های واسطه ای مانند قرار گرفتن آن روی لبه وجود داشته باشد، اما در حالت ایده آل می توان آنها را نادیده گرفت. 

در مقابل، یک کیوبیت می تواند در یک طیف پیوسته از حالت ها بین $\vert0\rangle$ و $\vert1\rangle$ وجود داشته باشد - تا زمانی که مشاهده شود. بار دیگر تاکید می کنیم که وقتی یک کیوبیت اندازه گیری می شود، فقط یکی از اازمقادیر «0» یا «1» را به عنوان نتیجه اندازه گیری می دهد. به عنوان مثال، یک کیوبیت می تواند در حالت
 $\vert0\rangle + \vert1\rangle$ باشد، که به این معنی است که با احتمال 50/50 می تواند به عنوان 0 یا 1 اندازه گیری شود.


\begin{center}
	$\frac{1}{\sqrt{2}} \vert 0\rangle + \frac{1}{\sqrt{2}} \vert 1\rangle$
\end{center}

حالت $\vert+\rangle$ حالتی از کیوبیت است که با یک بردار 2 بعدی واحد نشان داده می شود. این حالت، زمانی که اندازه گیری شود، نتیجه 0 را 50 درصد از زمان و نتیجه 1 را 50 درصد از زمان می دهد. این حالت را می توان به عنوان یک ترکیب خطی از دو حالت پایه $\vert0\rangle$ و $\vert 1 \rangle$ در نظر گرفت.

این حالت به دلیل عجیب بودنش جالب است. حالت های پایه $\vert 0 \rangle$ و $\vert 1 \rangle$ تنها حالاتی هستند که می توانند به طور مستقیم مشاهده شوند. حالت $\vert + \rangle$ ، با این حال، یک حالت ترکیبی است که به طور مستقیم قابل مشاهده نیست. تنها زمانی می توان آن را مشاهده کرد که اندازه گیری شود.

با وجود غیرقابل مشاهده بودن، حالت $\vert + \rangle$ واقعی است. وجود آن توسط آزمایشات به طور گسترده ای تأیید شده است. همچنین می توان از آن برای انجام محاسبات کوانتومی استفاده کرد.

در آینده، ممکن است حالت $\vert + \rangle$ برای اهداف مختلف دیگری نیز استفاده شود. به عنوان مثال، می تواند برای ذخیره اطلاعات یا برای ایجاد ارتباطات امن استفاده شود.
\pagebreak
\subsection{کیوبیت‌های چندتایی}

\begin{latin}
	Hilbert space is a big place.\\
	\hspace{2cm}– Carlton Caves
\end{latin}




فرض کنید دو کیوبیت داریم. اگر این دو بیت کلاسیک بودند، چهار حالت ممکن وجود داشت: 00، 01، 10 و 11. به همین ترتیب، یک سیستم دو کیوبیتی دارای چهار حالت محاسباتی است که با $\vert 00 \rangle, \vert 01 \rangle, \vert 10 \rangle, \vert 11 \rangle$ نشان داده می شود. یک جفت کیوبیت همچنین می تواند در برهمنهی این چهار حالت وجود داشته باشد.
بنابراین حالت کوانتومی دو کیوبیت با اختصاص یک عدد مختلط - گاهی اوقات به عنوان یک دامنه شناخته می شود - به هر حالت محاسباتی، بیان می‌شود. بردار حالت توصیف کننده دو کیوبیت به شکل زیر است:
\begin{center}
	$\vert \Psi \rangle = a \vert00\rangle + b \vert01\rangle + c \vert10\rangle + d \vert11\rangle $
\end{center}

جایی که a، b، c و d دامنه های چهار حالت را نشان می دهند. دامنه ها می توانند هر عدد مختلطی باشند، اما معمولاً به گونه ای نرمال می شوند که مجموع آنها برابر 1 باشد. این بدان معناست که بردار حالت یک حالت کوانتومی معتبر را نشان می دهد و کیوبیت ها به طور مساوی احتمال اندازه گیری در هر یک از چهار حالت محاسباتی را دارند.

به عنوان مثال، بردار حالت:

\begin{center}
$\vert \Psi \rangle = \frac{1}{\sqrt{2}} \vert00\rangle + \frac{1}{\sqrt{2}} \vert11\rangle$	
\end{center}

یک سیستم دو کیوبیتی را نشان می دهد که در یک برهمنهی مساوی از حالت های 00 و 11 است. این بدان معناست که کیوبیت ها به طور مساوی احتمال اندازه گیری در حالت 00 یا 11 را دارند.

بردار حالت یک سیستم دو کیوبیتی را می توان برای محاسبه احتمال اندازه گیری کیوبیت ها در هر یک از چهار حالت محاسباتی استفاده کرد. به عنوان مثال، احتمال اندازه گیری کیوبیت ها در حالت 00 با فرمول زیر داده می شود:
\begin{center}
$P(\vert00\rangle) = \vert a\vert^ 2 = \frac{1}{2}$
\end{center}


احتمال اندازه گیری کیوبیت ها در هر حالت دیگر را می توان به روشی مشابه محاسبه کرد.

نتیجه اندازه گیری $x (= 00, 01, 10 , 11)$ با احتمال $\vert a_{x} \vert ^ 2$ رخ می دهد، با حالت کیوبیت ها پس از اندازه گیری$\vert x \rangle$ . این بدان معناست که اگر ما یک سیستم دو کیوبیتی را در حالت
$\vert \Psi \rangle = a \vert00\rangle + b \vert01\rangle + c \vert10\rangle + d \vert11\rangle $ داشته باشیم، و اگر ما اولین کیوبیت را اندازه گیری کنیم، احتمال اینکه 0 را اندازه گیری کنیم برابر $\vert a \rangle^2 + \vert b \rangle ^ 2$ خواهد بود. در این حالت، حالت کیوبیت ها پس از اندازه گیری $\vert 0 \rangle$ خواهد بود.




\begin{center}
	$\left|\psi^{\prime}\right\rangle=\frac{\alpha_{00}|00\rangle+\alpha_{01}|01\rangle}{\sqrt{\left|\alpha_{00}\right|^2+\left|\alpha_{01}\right|^2}}$
\end{center}
توجه داشته باشید که حالت پس از اندازه گیری با عامل $\sqrt{|{α}_{00}|^2 + |{α}_{01}|^2}$ نرمال می شود
تا همچنان شرط نرمال سازی را، درست همانطور که برای یک
حالت کوانتومی معتبر انتظار می‌‌رود، برآورده کند. این بدان معناست که حالت پس از اندازه گیری به گونه ای تغییر می کند که احتمالات آن جمع شده و برابر 1 شود.

\section{اندازه‌گیری در فضای هیلبرت}
\subsection{bloch vector}
% https://g.co/bard/share/2972a6a63b24
ما تاکنون اندازه گیری های کوانتومی یک کیوبیت در حالت $\alpha\vert0\rangle + \beta\vert1\rangle$را به عنوان نتیجه 0 یا 1 توصیف کرده ایم که کیوبیت را در حالت $\vert 0 \rangle$ یا $\vert 1 \rangle$مربوطه باقی می گذارد، با احتمالات $\vert\alpha\vert ^ 2$ و $\vert\beta\vert ^ 2$. در حقیقت، مکانیک کوانتوم به اندازه کافی انعطاف پذیری در کلاس اندازه گیری هایی که می توان انجام داد، اگرچه مطمئناً به اندازه کافی نیست که $\alpha$ و $\beta$ را از یک اندازه گیری واحد بازیابی کند!

توجه داشته باشید که $\vert 0 \rangle$ و $\vert 1 \rangle$ فقط یکی از بسیاری از انتخاب های ممکن برای پایه های حالت برای یک کیوبیت هستند. یک انتخاب دیگر مجموعه به شرح زیر است:
\begin{center}
$\vert+\rangle  \equiv (\vert0\rangle + \vert1\rangle)\sqrt{2}$
$\vert-\rangle \equiv  (\vert0\rangle - \vert1\rangle)\sqrt{2}$
\end{center}
 
 
 یک حالت دلخواه $\vert\Psi\rangle = \alpha0\rangle + \beta1\rangle$ را می توان با استفاده از حالت های $\vert + \rangle$ و $\vert - \rangle$ بازنویسی کرد:
\begin{center}
	$\vert\Psi\rangle = \alpha \vert0\rangle + \beta\vert1\rangle = \frac{(\alpha + \beta)}{2}\vert+\rangle + \frac{(\alpha - \beta)}{2}\vert-\rangle$
\end{center}


در این بیان، $\vert+\rangle$و $\vert-\rangle$ به عنوان پایه های "پایه +" و "پایه -" شناخته می شوند. اندازه گیری در پایه + یا پایه - یک کیوبیت را در حالت $\vert+\rangle$ یا $\vert-\rangle$قرار می دهد.
 

اندازه گیری در پایه های دیگر به غیر از پایه محاسباتی یک ابزار قدرتمند در محاسبات کوانتومی است. این امکان را می دهد تا ما در حالت های کوانتومی که در پایگاه محاسباتی قابل اندازه گیری نیستند، اندازه گیری کنیم. این امکان را می دهد تا ما از روش های محاسباتی جدیدی استفاده کنیم که در محاسبات کلاسیک غیرممکن است.

در واقع، این امکان وجود دارد که حالت های $\vert+\rangle$ و $\vert-\rangle$ را به گونه ای که گویی آنها حالت های پایه محاسباتی هستند، در نظر بگیریم و با توجه به این پایه جدید اندازه گیری کنیم. طبیعی است که اندازه گیری با توجه به پایه $\vert-\rangle$، $\vert+\rangle$ منجر به نتیجه "+" با احتمال $\frac{\vert\alpha + \beta\vert}{2}^2$ و نتیجه "-" با احتمال $\frac{\vert\alpha - \beta\vert}{2}^2$ می شود.


در این بیان، $\vert+\rangle$ و $\vert-\rangle$ به عنوان "پایه +" و "پایه -" شناخته می شوند. اندازه گیری در پایه + یا پایه - یک کیوبیت را در حالت $\vert+\rangle$ یا $\vert-\rangle$ قرار می دهد.

به طور کلی تر، با توجه به هر دو پایه حالت $\vert a\rangle$ و $\vert b\rangle$ برای یک کیوبیت، می توان هر حالت دلخواهی را به عنوان یک ترکیب خطی $\alpha \vert a\rangle + \beta \vert b\rangle$ از آن حالات بیان کرد. علاوه بر این، اگر این حالات  متعامد باشند، می توان با توجه به پایه $\vert a\rangle$، $\vert b\rangle$ اندازه گیری کرد، که نتیجه a با احتمال $\vert\alpha\vert ^2$ و b با احتمال $\vert\beta\vert ^2$ قابل رخداد است.

با توجه به بیان احتمالاتی مفهوم کیوبیت و نرمال بودن مقادیر احتمالات و متعامد بودن پایه‌ها، لازم است تا $\vert\alpha\vert ^ 2 + \vert\alpha\vert ^ 2 = 1$ باشد همانطور که برای احتمالات انتظار می رود. به طور مشابه، در اصل می توان یک سیستم کوانتومی از بسیاری از کیوبیت ها را با توجه به یک پایه متعامد دلخواه اندازه گیری کرد.

اندازه گیری در پایه های دیگر یک ابزار قدرتمند در محاسبات کوانتومی است. این امکان را می دهد تا ما در حالت های کوانتومی که در پایگاه محاسباتی قابل اندازه گیری نیستند، اندازه گیری کنیم. این امکان را می دهد تا ما از روش های محاسباتی جدیدی استفاده کنیم که در محاسبات کلاسیک غیرممکن است.

دلایل زیادی برای استفاده از این مدل توسعه‌یافته برای اندازه گیری های کوانتومی وجود دارد، اما در نهایت بهترین دلیل این است: این مدل به ما امکان توصیف نتایج تجربی مشاهده شده نظیر نتایج آزمایش اشترن-گرلاخ\footnote{Stern–Gerlach} را می دهد.

\newpage
\section{گیت‌های کوانتومی}
گیت‌های کوانتومی\footnote{\lr{Quantum Gates}} یکی از اولین و مهم‌ترین اجزای‌ مدار‌های کوانتومی ‌می‌باشند. این گیت‌ها عملگر‌هایی با قابلیت اثر‌گذاری روی کیوبیت‌ها می‌باشند. با اعمال یک گیت کوانتومی بر روی یک یا چند کیوبیت، می‌توان تغییرات مدنظر خود را روی کیوبیت اعمال کرد. با کمک این گیت‌ها می‌توان باعث برهم‌نهی کوانتومی یا رمز‌گذاری داده در داخل یک یا چند کیوبیت شد.

\subsection{انواع گیت کوانتومی}
گیت‌های کوانتومی، دارای انواع مختلف گوناگونی می‌باشند. به طور کلی گیت‌های کوانتومی، عملگر‌هایی یکه و بازگشت‌پذیر می‌باشند. به‌طور کلی گیت‌های‌کوانتومی متناسب با تعداد کیوبیت‌هایی که از آنها اثر‌میگیرند؛ دسته‌بندی می‌کنیم. در این گفتار به گیت‌های تک کیوبیتی و دو کیوبیتی می‌پردازیم.
\subsection*{گیت هادامارد}
مهم‌ترین گیت کوانتومی،‌گیت هادامارد\footnote{\lr{Hadamard gate}} است. با اعمال اثر این گیت روی یک کیوبیت، آن کیوبیت به‌یک حالت برهم نهی‌ کوانتومی‌ گذار‌ می‌کند. به عبارت دیگر هر یک از زیرحالات این حالت برهم‌نهی، با احتمال یکسانی قابل رخ دادن‌ هستند. 
\vspace{1cm}
$$
H\ket{0} = \frac{1}{\sqrt{2}} (\ket{0} + \ket{1})
\hspace{2cm}
H\ket{1} = \frac{1}{\sqrt{2}} (\ket{0} - \ket{1})
$$
\vspace{1cm}

این گیت کوانتومی‌ به صورت خطی روی یک دسته‌ کت اثر مي‌کند. نمایش ماتریسی این گیت‌کوانتومی به شرح زیر است:
\begin{center}
	\[
	H = \frac{1}{\sqrt{2}}
	\begin{pmatrix}
		1 & 1 \\
		1 & -1
	\end{pmatrix}
	\]
\end{center}





\begin{align*}
	H |0\rangle = \frac{1}{\sqrt{2}} \begin{pmatrix} 1 & 1 \\ 1 & -1 \end{pmatrix} \begin{pmatrix} 1 \\ 0 \end{pmatrix} \
	= \frac{1}{\sqrt{2}} \begin{pmatrix} 1 \\ 1 \end{pmatrix} \
	= \frac{1}{\sqrt{2}} (|0\rangle + |1\rangle)
\end{align*}

\begin{align*}
	H |1\rangle = \frac{1}{\sqrt{2}} \begin{pmatrix} 1 & 1 \\ 1 & -1 \end{pmatrix} \begin{pmatrix} 0 \\ 1 \end{pmatrix} \
	= \frac{1}{\sqrt{2}} \begin{pmatrix} 1 \\ -1 \end{pmatrix} \
	= \frac{1}{\sqrt{2}} (|0\rangle - |1\rangle)
\end{align*}

این گیت کوانتومی، یک گیت بازگشت‌پذیر است؛ یعنی اگر این گیت روی یک حالت کوانتومی اثر کند؛‌ می‌تواند آن را از حالت برهمنهی خارج کند. 

برای اعمال این گیت کوانتومی، فقط به یک کیوبیت نیاز داریم. به اصطلاح این گیت،‌
یک \lr{Single-Qubit Quantum gate} می‌باشد.

نمایش این گیت‌کوانتومی در مدار با علامت زیر است:
\Qcircuit @C=1em @R=1.2em {
	& & \qw & \gate{H} & \qw \\
	& & & & \\
}

\subsection*{گیت NOT}
گیت کوانتومی $U_{NOT}$ یک عملگر یکه و بازگشت‌پذیر است. 

\begin{center}
	$U_{NOT}\vert0\rangle = \vert1\rangle$\\
	$U_{NOT}\vert1\rangle = \vert0\rangle$
\end{center}


\subsubsection{نمایش ماتریسی گیت NOT}

می بینید که اگر فقط پایه را در نظر بگیریم، دقیقاً مانند دروازه کلاسیک NOT است. نمایش ماتریسی این گیت به شکل زیر است:
\begin{center}
	$U_{NOT} =
	 \begin{pmatrix}
	 	0 & 1\\
	 	1 & 0
		\end{pmatrix}$
\end{center}
\pagebreak

از آنجایی که یک حالت کیوبیت در فضای هیلبرت دو بعدی تعریف شده است. ابعاد ماتریس باید 2 × 2 باشد. از این رو می‌توان نوشت:
$$
\begin{aligned}
	U_{N O T}|0\rangle & =\left(\begin{array}{ll}
		0 & 1 \\
		1 & 0
	\end{array}\right)|0\rangle \\
	& =\left(\begin{array}{ll}
		0 & 1 \\
		1 & 0
	\end{array}\right)\left(\begin{array}{l}
		1 \\
		0
	\end{array}\right)=\left(\begin{array}{l}
		0 \\
		1
	\end{array}\right)=|1\rangle
\end{aligned}
$$\\

$$
\begin{aligned}
	\boldsymbol{U}_{\text {NOT }}|1\rangle & =\left(\begin{array}{ll}
		0 & 1 \\
		1 & 0
	\end{array}\right)|1\rangle \\
	& =\left(\begin{array}{ll}
		0 & 1 \\
		1 & 0
	\end{array}\right)\left(\begin{array}{l}
		0 \\
		1
	\end{array}\right)=\left(\begin{array}{l}
		1 \\
		0
	\end{array}\right)=|0\rangle
\end{aligned}
$$

\subsubsection{مدار NOT و خصوصیات آن}
با اعمال گیت $U_{NOT}$ به حالت $\vert \Psi \rangle = \alpha \vert 0 \rangle + \beta \vert 1\rangle$ خواهیم داشت:
 
\begin{center}
$$	\begin{aligned}
		\boldsymbol{U}_{\text {NOT }}|\Psi\rangle & =\left(\begin{array}{ll}
			0 & 1 \\
			1 & 0
		\end{array}\right)|\Psi\rangle \\
		& =\left(\begin{array}{ll}
			0 & 1 \\
			1 & 0
		\end{array}\right)\left(\begin{array}{l}
			\alpha \\
			\beta
		\end{array}\right)=\left(\begin{array}{l}
			\beta \\
			\alpha
		\end{array}\right)=\beta|0\rangle+\alpha|1\rangle
	\end{aligned}$$
\end{center}

علاوه بر دروازه \lr{NOT} یک کیوبیتی، می‌توانیم دو کیوبیتی نیز بسازیم.
می‌بایست حاصل ضرب تانسوری دو دروازه \lr{NOT} یک کیوبیتی را محاسبه کنیم؛ تا یک دروازه \lr{NOT} دو کیوبیتی بدست آوریم زیرا یک بردار دو کیوبیتی در فضای$\mathbb{C}^4$ قرار دارد.
که حاصل ضرب تانسور دو فضای$\mathbb{C}^2$است. از این رو:
\begin{center}
	$$\begin{aligned}
		U_{N O T_2} & =U_{N O T} \otimes U_{N O T}=\left(\begin{array}{ll}
			0 & 1 \\
			1 & 0
		\end{array}\right) \otimes\left(\begin{array}{ll}
			0 & 1 \\
			1 & 0
		\end{array}\right) \\
		& =\left(\begin{array}{ll}
			0\left(\begin{array}{ll}
				0 & 1 \\
				1 & 0
			\end{array}\right) 1\left(\begin{array}{ll}
				0 & 1 \\
				1 & 0
			\end{array}\right) \\
			1\left(\begin{array}{ll}
				0 & 1 \\
				1 & 0
			\end{array}\right) 0\left(\begin{array}{ll}
				0 & 1 \\
				1 & 0
			\end{array}\right)
		\end{array}\right)=\left(\begin{array}{llll}
			0 & 0 & 0 & 1 \\
			0 & 0 & 1 & 0 \\
			0 & 1 & 0 & 0 \\
			1 & 0 & 0 & 0
		\end{array}\right)
	\end{aligned}$$
\end{center}
نمایش یک گیت \lr{NOT} به صورت زیر است:

\Qcircuit @C=1em @R=.7em {
	& \qw & \gate{X} & \qw & \qw
}

با اعمال گیت دوکیوبیتی \lr{NOT} به یک سیستم دو کیوبیتی نظیر 

$\vert\Psi \rangle = \alpha \vert 00\rangle + \beta \vert 01\rangle + \gamma \vert 10\rangle + \delta \vert 11\rangle$ خواهیم داشت: 
 
\begin{center}
	$$\boldsymbol{U}_{\boldsymbol{N} \boldsymbol{O} \boldsymbol{T}_2}|\Psi\rangle=\left(\begin{array}{llll}
		0 & 0 & 0 & 1 \\
		0 & 0 & 1 & 0 \\
		0 & 1 & 0 & 0 \\
		1 & 0 & 0 & 0
	\end{array}\right)\left(\begin{array}{l}
		\alpha \\
		\beta \\
		\gamma \\
		\delta
	\end{array}\right)=\left(\begin{array}{l}
		\delta \\
		\gamma \\
		\beta \\
		\alpha
	\end{array}\right)$$
\end{center}


\subsection*{گیت CNOT}
% https://www.youtube.com/watch?v=0ECbWBBbglw
% https://www.youtube.com/watch?v=emHhNFf5AVM

گیت کوانتومی \lr{CNOT}\footnote{ controlled-NOT gate or controlled-X gate}، به عنوان گیت منطقی نیز یاد می‌شود. این گیت کوانتومی معادل گیت \lr{NOT} کلاسیک می‌باشد.
به‌طور معمول، برای اعمال اثر این گیت کوانتومی نیاز به دو کیوبیت داریم. این گیت کوانتومی فقط و فقط در مواقعی که «کیوبیت کنترل\footnote{Controled Qubit}» دارای مقدار $\vert 1 \rangle$ باشد، باعث تغییر وضعیت «کیوبیت هدف\footnote{Target Qubit}» می‌شود.\\

کیوبیت کنترلی: کیوبیتی است که عملکرد کیوبیت دیگری به نام کیوبیت هدف را کنترل می کند. کیوبیت کنترل تعیین می کند که آیا کیوبیت هدف برگردانده شود یا خیر. اگر کیوبیت کنترل در حالت $\vert 0 \rangle$ باشد، کیوبیت هدف بدون تغییر باقی می‌ماند. اگر کیوبیت کنترل در حالت $\vert 1 \rangle$ باشد، کیوبیت هدف برگردانده می شود.\\

کیوبیت هدف: همان کیوبیتی است که توسط کیوبیت کنترل بر روی آن عمل می شود. بسته به وضعیت کیوبیت کنترل، کیوبیت هدف را می توان برگرداند یا بی‌تغییر رها کند.\\


\begin{center}
	$U_{CNOT}\vert ab\rangle = \vert a a\oplus \rangle b$
\end{center}


 $\vert ab \rangle$ چیست؟
  این یک بردار در فضای $\mathbb{C}^4$ و a و b دارای مقدار 0 یا 1 هستند.  $\vert ab \rangle$ می تواند هر یک از حالت های پایه(یعنی  $\vert 00 \rangle, \vert 01 \rangle, \vert 10 \rangle, \vert 11 \rangle$) باشد.
 
  
  
   این معادله به ما می گوید؛ که پس از اعمال گیت \lr{CNOT} به یکی از پایه‌ها، آن حالت به حالت پایه دیگری تبدیل خواهد شد. که این حالت برابر $\vert a\hspace{0.1cm} a \oplus b\rangle$ می‌باشد. به‌طوری که:
   
   \begin{center}
   	$\vert a\hspace{0.1cm}a\oplus b\rangle = \vert a \rangle \otimes \vert a \oplus b \rangle$ 
   \end{center}

 بعد از عمل، عدد اول همچنان a است، اما عدد دوم به $a \oplus b$ تبدیل می شود، جایی که $\oplus$ عملیات منطقی کلاسیک انحصاری یا (\lr{XOR}) است. توجه کنید که:
 \begin{center}
 	$0 \oplus 0 = 0$\\
 	$0 \oplus 1 = 1$\\
 	$1 \oplus 0 = 1$\\
 	$1 \oplus 1 = 0$\\
 \end{center}
 
\begin{center}
	$U_{CNOT}|00\rangle=|0,0 \oplus 0\rangle=|0,0\rangle=|00\rangle$ \\
	$U_{CNOT}|01\rangle=|0,0 \oplus 1\rangle=|0,1\rangle=|01\rangle$ \\
	$U_{CNOT}|10\rangle=|1,1 \oplus 0\rangle=|1,1\rangle=|11\rangle$\\
	$U_{CNOT}|11\rangle=|1,1 \oplus 1\rangle=|1,1\rangle=|10\rangle$
\end{center}

خلاصه‌ای از عملکرد این عملگر به شرح زیر است:\\

\begin{latin}
\begin{tabular}{ccccc}
	&&&&\\
	|A & B$\rangle$ &	&  |A &B $\oplus$ A $\rangle$  \\
	|control$\rangle$ & |target$\rangle$ & Effect CNOT Gate &|control$\rangle$ & |target$\rangle$ \\
	------- & -------- & -------- & -------- & --------  \\
	|0$\rangle$ & |0$\rangle$ & $\Longrightarrow$ &|0$\rangle$ & |0$\rangle$ \\
	|0$\rangle$ & |1$\rangle$ & $\Longrightarrow$ &|0$\rangle$ & |1$\rangle$ \\
	|1$\rangle$ & |0$\rangle$ & $\Longrightarrow$ &|1$\rangle$ & |1$\rangle$ \\
	|1$\rangle$ & |1$\rangle$ & $\Longrightarrow$ &|1$\rangle$ & |0$\rangle$
\end{tabular}
\end{latin}


نمایش ماتریسی این گیت کوانتومی به شکل زیر است:
\begin{figure}[ht]
	\centering
	\includegraphics[width=\linewidth]{Matrix-representation-and-quantum-circuit-of-CNOT-gate.png}
	\caption{نمایش ماتریسی و نمایش گیت کوانتومی \lr{CNOT}}
	\label{CNOT}
\end{figure}


در شکل بالا گیت \lr{CNOT} در مدار کوانتومی به تصویر درآمده است. کیوبیت کنترل شده  حالت $\vert a \rangle$ و کیوبیت هدف حالت $\vert b \rangle$ می‌باشد.


با اعمال این عملگر به حالت $\vert 10\rangle$ داریم:
$$
\text{CNOT} \ket{10} = \begin{pmatrix}
	1 & 0 & 0 & 0 \\
	0 & 1 & 0 & 0 \\
	0 & 0 & 0 & 1 \\
	0 & 0 & 1 & 0
\end{pmatrix} \begin{pmatrix}
	0 \\
	0 \\
	1 \\
	0
\end{pmatrix} = \begin{pmatrix}
	0 \\
	0 \\
	0 \\
	1
\end{pmatrix} = \ket{11}
$$\\

این فرآیند به صورت معکوس نیز قابل رخ دادن است:\\

$$
\text{CNOT} \ket{11} = \begin{pmatrix}
	1 & 0 & 0 & 0 \\
	0 & 1 & 0 & 0 \\
	0 & 0 & 0 & 1 \\
	0 & 0 & 1 & 0
\end{pmatrix} \begin{pmatrix}
	1 \\
	1 \\
	0 \\
	0
\end{pmatrix} = \begin{pmatrix}
	0 \\
	1 \\
	0 \\
	0
\end{pmatrix} = \ket{10}
$$
\vspace{2cm}


از این گیت کوانتومی‌، برای بسیاری مدار‌ها و شبیه‌سازی‌های کوانتومی، از جمله تلپورت، درهمتنیدگی و ...،استفاده می‌شود.

\subsection*{گیت تغییر فاز}
گیت تغییر فاز که به آن گیت P یا گیت فاز نیز می گویند، یک کیوبیت است.
دروازه ای که فاز نسبی را بین دو بردار پایه جابجا می کند. به عنوان زیر تعریف شده است:
\begin{center}
	$$\begin{aligned}
		& \boldsymbol{U}_{\boldsymbol{PS}, \boldsymbol{\phi}}|0\rangle=|0\rangle \\
		& \boldsymbol{U}_{\boldsymbol{PS},  \boldsymbol{\phi}}|1\rangle=e^{i \Phi}|1\rangle
	\end{aligned}$$
\end{center}


که در آن $\Phi$ فاز و $e^{i\Phi}$ عامل فاز است. 
زمانی که گیت تغییر فاز بر حالت $\vert 0 \rangle$  اثرکند، هیچ تغییری حاصل نمی‌شود. اما با اعمال این گیت به بردار حالت $\vert 1 \rangle$، یک فاز اضافه به بردار افزوده خواهد شد. برای افزدون فاز اضافه  باید یک عامل نظیر  $e^{i\Phi}$ در بردارد حالت ضرب شود. برای درک بهتر مطلب باید به حالت ماتریسی این گیت رجوع کنیم.





\subsubsection{نمایش ماتریسی گیت تغییر فاز}

نمایش ماتریسی گیت تغییر فاز $\boldsymbol{U}_{\boldsymbol{P} \boldsymbol{S}, \boldsymbol{\Phi}}$ به صورت زیر است:

\begin{center}
	$$\boldsymbol{U}_{\boldsymbol{P} \boldsymbol{S}, \boldsymbol{\Phi}}=\left(\begin{array}{cc}
		1 & 0 \\
		0 & e^{i \Phi}
	\end{array}\right)$$
\end{center}
با استفاده از روابط بالا می‌توان  به راحتی اثبات کرد: 
\begin{center}
	$$U_{P S, \boldsymbol{\phi}}|1\rangle=\left(\begin{array}{cc}
		1 & 0 \\
		0 & e^{i \Phi}
	\end{array}\right)\left(\begin{array}{l}
		0 \\
		1
	\end{array}\right)=\left(\begin{array}{c}
		0 \\
		e^{i \Phi}
	\end{array}\right)=e^{i \Phi}\left(\begin{array}{l}
		0 \\
		1
	\end{array}\right)=e^{i \Phi}|1\rangle$$
\end{center}




\subsubsection{گیت تغییر فاز در مدار و خصوصیات آن}
درحالت کلی برای سیستم های تک کیوبیتی می‌توان نوشت:

\begin{center}
	$$\begin{aligned}
		\boldsymbol{U}_{\boldsymbol{P S}, \boldsymbol{\Phi}}|\Psi\rangle & =\left(\begin{array}{cc}
			1 & 0 \\
			0 & e^{i \Phi}
		\end{array}\right)\left(\begin{array}{l}
			\alpha \\
			\beta
		\end{array}\right)=\left(\begin{array}{c}
			\alpha \\
			e^{i \Phi} \beta
		\end{array}\right) \\
		& =\alpha|0\rangle+e^{i \Phi} \beta|1\rangle
	\end{aligned}$$
\end{center}
در بردار اصلی، $\alpha$ و $\beta$ اعداد مختلط می‌باشند؛ به همین دلیل این اعداد را می‌توان به به عنوان یک ضریب فاز معرفی کرد:

 \begin{center}
 	$\alpha = e^{i\theta_{1} \vert\alpha\vert }, \beta = e^{i\theta_{2} \vert\beta\vert}$
 \end{center}
که در آن $\theta1$ و $\theta2$ به ترتیب فازهای $\alpha$ و $\beta$ هستند.

بنابراین، اختلاف فاز آنها θ1 - θ2 است. با توجه به رفتارگیت تغییر فاز، تغییر فاز $\alpha$ صفر می‌باشد؛‌اما فاز β به Φ + θ2 تغییر می کند. بنابراین، خویشاوندان آنها فاز با Φ تغییر می کند. 

\begin{center}
	 This also explains why the phase shift is only applied to the
	second basis state to be meaningful. If it is applied to both, there will be no change
	in the phase difference.
\end{center}

 (1 0
0 e iπ
)
=
(1 0
0-1
)
زیرا e iπ = cos π + i sin π = -1.



 این ماتریس پائولی $ \sigma_{z}$ می‌باشد؛ که به آن Z-gate نیز می گویند. گیت تغییر فاز، فاز نسبی $vert 0 \rangle$و |1〉 را با Φ تغییر می دهد. و به این ترتیب گیت $Z$ فاز نسبی را با $\pi$ تغییر می دهد. 
 
 
 
\subsection{Swap gate}
یک گیت SWAP اعداد را در حالت های پایه یک رجیستر 2 کیوبیتی جابجا می کند. اگر فقط حالت های پایه را در نظر بگیریم، معادل مبادله حالات دو الکترون است. این گیت به صورت  زیر تعریف می‌شود:
\begin{center}
	$U_{SWAP}   \vert ab\rangle = \vert ba \rangle$
\end{center}
که در آن a و b فقط اعداد کیوبیت اول و دوم در پایه هستند. a و b می‌توانند مقادیر 0 و 1 را اتخاذ کنند. بنابرین می‌توان نوشت:
\begin{center}
	$$\begin{aligned}
		& U_{S W A P}|00\rangle=|00\rangle \\
		& U_{S W A P}|01\rangle=|10\rangle \\
		& U_{S W A P}|10\rangle=|01\rangle \\
		& U_{S W A P}|11\rangle=|11\rangle
	\end{aligned}$$
\end{center}




بنابراین، تنها $\vert 01 \rangle$ و $\vert 10 \rangle$ تحت SWAP (به یکدیگر) تغییر می کنند.
اعمال این گیت برای حالات $\vert 00 \rangle$ و $\vert 11 \rangle$، بدون اثر است؛ زیرا "0" و "0" یا "1" و "1" تعویض می‌شوند؛ که پیامد خاصی در پی ندارد.


\subsubsection{نمایش ماتریسی گیت swap}
نمایش ماتریسی $U_{S W A P}$ به شکل زیر است:
\begin{center}
	$$U_{S W A P}=\left(\begin{array}{llll}
		1 & 0 & 0 & 0 \\
		0 & 0 & 1 & 0 \\
		0 & 1 & 0 & 0 \\
		0 & 0 & 0 & 1
	\end{array}\right)$$
\end{center}
\
 از آنجایی که این یک گیت 2 کیوبیتی برای عملیات است یک بردار در فضای چهار بعدی $\mathbb{c}^4$، ماتریس (اپراتور) باید 4 × 4 باشد. علاوه بر این، فقط حالت پایه دوم و سوم را جابجا می کند. بنابراین، تنها ردیف دوم و سوم با ماتریس یکه متفاوت هستند. به طور مثال:
\begin{center}
	$$U_{S W A P}|10\rangle=\left(\begin{array}{llll}
		1 & 0 & 0 & 0 \\
		0 & 0 & 1 & 0 \\
		0 & 1 & 0 & 0 \\
		0 & 0 & 0 & 1
	\end{array}\right)\left(\begin{array}{l}
		0 \\
		0 \\
		1 \\
		0
	\end{array}\right)=\left(\begin{array}{l}
		0 \\
		1 \\
		0 \\
		0
	\end{array}\right)=|01\rangle$$
\end{center}
این رابطه‌ بصورت معکوس نیز برقرارست.

\subsubsection{گیت swap در مدار و خصوصیات آن}
به طور کلی، برای یک بردار 2 کیوبیتی، نظیر $\vert \Psi \rangle=\alpha|00\rangle+\gamma|01\rangle+\beta|10\rangle+\delta|11\rangle$ داریم:

\begin{center}
	$$\begin{aligned}
		\boldsymbol{U}_{\boldsymbol{SWAP}}|\Psi\rangle & =\left(\begin{array}{llll}
			1 & 0 & 0 & 0 \\
			0 & 0 & 1 & 0 \\
			0 & 1 & 0 & 0 \\
			0 & 0 & 0 & 1
		\end{array}\right)\left(\begin{array}{l}
			\alpha \\
			\beta \\
			\gamma \\
			\delta
		\end{array}\right)=\left(\begin{array}{l}
			\alpha \\
			\gamma \\
			\beta \\
			\delta
		\end{array}\right) \\
		\vspace{2cm}
		& =\alpha|00\rangle+\gamma|01\rangle+\beta|10\rangle+\delta|11\rangle
	\end{aligned}$$
\end{center}





\begin{figure}[ht]
	\centering
	\includegraphics[width=50mm,scale=0.5]{swap.png}
	\caption{ نمایش گیت کوانتومی \lr{SWAP}}
	\label{SWAP}
\end{figure}



همانطور که انتظار می رود، ضرایب حالت های پایه دوم و سوم با هم مبادله می شوند.
\newpage
\section{مدار‌های کوانتومی}
مدار‌های ‌کوانتومی\footnote{quantum circuit}، یک دسته از گیت ها‌ی کوانتومی،که با یک توالی بخصوص قرار گرفته اند، ‌می‌باشند. این کیوبیت ها، با توالی یاد شده، روی یک یا چند دسته کیوبیت، اثر داده ‌می‌شوند. 

مدار‌های کوانتومی، یکی از اولین مفهوم‌های بکار‌رفته برای تعریف کامپیوتر‌های کوانتومی‌ می‌باشند. برای تعریف و شبیه‌سازی هریک از پدیده‌ها و الگوریتم‌های کوانتومی، نیاز به پیاده‌سازی یک مدار به‌خصوص داریم.


مدارهای کوانتومی یک ابزار قدرتمند برای محاسبات کوانتومی هستند. آنها می توانند برای پیاده سازی طیف گسترده ای از الگوریتم های کوانتومی، از جمله الگوریتم شاور برای رمزگشایی اعداد صحیح و الگوریتم گروور برای جستجوی پایگاه داده های بدون ترتیب استفاده شوند.


\begin{figure}[htbp]
	\centering
	\includegraphics[width=\linewidth]{cnot in circuit.png}
	\caption{نمایش‌های مختلف گیت CNOT در مدار کوانتومی}
	\label{fig:my_image}
\end{figure}

یکی دیگر از عملگر‌های مهم در مدار کوانتومی، عملگر اندازه‌گیری است. که در شکل زیر نشان داده شده است. همانطور که پیش از این بیان شد؛ در بردار حالت $\vert \Psi \rangle = \alpha \vert 0 \rangle + \beta \vert 1 \rangle$ هنگام مشاهده به یک حالت کلاسیکی رمبش می‌کند؛‌احتمال اینکه به حالت $\vert 0 \rangle$رمبش کند، $\vert \alpha \vert ^2$, و احتمال اینکه به حالت $\vert 1 \rangle$ رمبش کند، $\vert \beta \vert ^2$، می‌باشد.

 
\begin{figure}[ht]
	\centering
	\includegraphics[width=\linewidth, scale=0.5]{measurment.png}
	\caption{نمایش‌های مختلف گیت CNOT در مدار کوانتومی}
	\label{fig:my_image}
\end{figure}


مدار‌های کوانتومی مدلی بسیار سودمند برای شبیه‌سازی‌ فرآیندهای کوانتومی‌ است. این فرآیندها محدود به عملیات‌های محاسباتی نخواهد شد؛ بلکه در اغلب حوزه‌‌ها اعم از ارتباطات، اختلالات کوانتومی\footnote{quantum noise} و ... بکار‌خواهدرفت.
\subsubsection{شباهت ها و تفاوت های مدارهای کلاسیک و کوانتومی}
مدارهای کوانتومی مشابه مدارهای کلاسیک هستند، اما از دروازه های کوانتومی به جای دروازه های منطقی کلاسیک استفاده می کنند. دروازه های کوانتومی عملیات قابل برگشت هستند که می توانند برای دستکاری حالت کوانتومی یک کیوبیت استفاده شوند.

\textbf{شباهت ها}

\begin{itemize}
	\item هر دو مدار کوانتومی و کلاسیک از یک دنباله عملیاتی تشکیل شده اند که به یک مجموعه داده اعمال می شوند.
	\item هر دو مدار را می توان به صورت گرافیکی با نماد مشابهی نشان داد.
	\item هر دو مدار می توانند برای پیاده سازی الگوریتم ها استفاده شوند.
\end{itemize}

\textbf{تفاوت ها}

\begin{itemize}
	\item مدارهای کوانتومی از کیوبیت ها، که معادل کوانتومی مفهوم بیت هستند، به عنوان واحد پایه داده خود استفاده می کنند. 
	\item مدارهای کلاسیک از بیت ها ، که بیت های کلاسیک هستند ، به عنوان واحد پایه داده خود استفاده می کنند.
	\item مدارهای کوانتومی از دروازه های کوانتومی ، که عملیات قابل برگشت هستند ، به عنوان عملیات پایه خود استفاده می کنند.
	\item  مدارهای کلاسیک از دروازه های منطقی ، که عملیات برگشت ناپذیر هستند ، به عنوان عملیات پایه خود استفاده می کنند.
	\item مدارهای کوانتومی می توانند خواص مکانیک کوانتوم را ، مانند برهمنهی و درهمتنیدگی ، برای انجام کارهایی که برای رایانه های کلاسیک غیرممکن است ، بهره مند شوند.
\end{itemize}



\begin{table}[ht]
	\centering
	\begin{tabular}{|p{100pt}|p{100pt}|p{100pt}|}
		\hline
		ویژگی & مدار کوانتومی &مدار کلاسیک \\
		\hline
		واحد پایه داده & کیوبیت & بیت \\
		\hline
	      عملیات پایه & دروازه های کوانتومی & دروازه های منطقی \\
		\hline
		برگشت پذیری & قابل برگشت & برگشت ناپذیر \\
		\hline
	\end{tabular}
	\caption{شباهت‌ها و تفاوت های مدارهای کلاسیک و کوانتومی}
\end{table}



\subsubsection{اجزای مدار‌های کوانتومی و سایز آن }
\paragraph{اندازه مدار کوانتومی}

اندازه یک مدار کوانتومی تعداد دروازه های موجود در مدار است. پیچیدگی یک الگوریتم کوانتومی اغلب با اندازه مدار کوانتومی مورد نیاز برای پیاده سازی آن اندازه گیری می شود.


\paragraph{کیوبیت}

کیوبیت ها واحد پایه اطلاعات در محاسبات کوانتومی هستند. آنها می توانند در یک برهنهی کوانتومی از دو حالت ، 0 و 1 باشند. این بدان معنی است که یک کیوبیت می تواند هم 0 و هم 1 باشد ، که یک ویژگی به نام برهمنی کوانتومی است. کیوبیت ها همچنین می توانند به هم متصل شوند ، که به این معنی است که حالت یک کیوبیت به حالت کیوبیت دیگر وابسته است.
\paragraph{دروازه}
دروازه ها عملیاتی هستند که روی کیوبیت ها اعمال می شوند. آنها می توانند برای ایجاد برهمنهی ، انجام چرخش ها و درهم تنیدگی کیوبیت ها استفاده شوند. انواع مختلفی از دروازه ها وجود دارد ، اما برخی از رایج ترین آنها شامل گیت هادامارد ، گیت CNOT و گیت توفولی است.

\paragraph{عملیات}

عملیات اقداماتی هستند که روی کیوبیت ها انجام می شوند. آنها می توانند اندازه گیری ها یا سایر اقدامات باشند. اندازه گیری ها برای رمبش حالت کوانتومی یک کیوبیت به یک مقدار قطعی ، 0 یا 1 استفاده می شود.



اجزای اساسی یک مدار کوانتومی کیوبیت ها ، دروازه ها و عملیات هستند. این اجزا برای ایجاد الگوریتم های کوانتومی استفاده می شوند که الگوریتم هایی هستند که فقط می توانند روی یک رایانه کوانتومی اجرا شوند. مدارهای کوانتومی یک ابزار قدرتمند برای محاسبات کوانتومی هستند و پتانسیل انقلابی در بسیاری از زمینه های مختلف ، از جمله رمزنگاری ، شیمی و یادگیری ماشین را دارند.

\subsection{نحوه‌ی نمایش مدار‌های کوانتومی}


مدارهای کوانتومی با استفاده از نماد گرافیکی مشابه نمودارهای مدار استفاده شده در محاسبات کلاسیک نوشته می شوند. محور افقی یک مدار کوانتومی زمان را نشان می دهد و محور عمودی کیوبیت ها را نشان می دهد. دروازه ها توسط جعبه ها نشان داده می شوند و خطوط بین جعبه ها نشان دهنده ارتباطات بین کیوبیت ها است.

\begin{figure}[ht]
	\centering
	\includegraphics[]{meas-firstcirc.png}
	\caption{مدارکوانتومی شبیه‌سازی شده برای آزمایش درهمتنیدگی}
\end{figure}


\newpage
\chapter{برنامه‌نویسی کوانتومی}
برنامه نویسی کوانتومی فرآیند طراحی و پیاده‌سازی دنباله هایی از دستورالعمل هایی موسوم مدارهای کوانتومی می‌باشد، با استفاده از گیت ها، سوئیچ ها و عملگرها برای دستکاری وضعیت کوانتومی یک کیوبیت به پردازش مسائل می‌پردازیم.

مدارهای کوانتومی یک نمایش گرافیکی از الگوریتم های کوانتومی هستند، این الگوریتم هایی فقط روی یک کامپیوتر کوانتومی قابل اجرا هستند.

برنامه نویسی کوانتومی یک زمینه نسبتاً جدید است و تعدادی زبان برنامه نویسی کوانتومی مختلف در دسترس است. برخی از محبوب ترین زبان های برنامه نویسی کوانتومی عبارتند از \lr{Qiskit}، \lr{Cirq} و \lr{Quil}.

برنامه نویسی کوانتومی یک زمینه پیچیده و چالش برانگیز است، اما این پتانسیل را دارد که در بسیاری از زمینه های مختلف از جمله رمزنگاری، شیمی و یادگیری ماشین انقلابی ایجاد کند. با قدرتمندتر شدن کامپیوترهای کوانتومی، برنامه نویسی کوانتومی اهمیت فزاینده ای پیدا خواهد کرد.
\section{کامپیوترهای کوانتومی در مقابل کامپیوترهای کلاسیک}



کامپیوترهای کوانتومی و کامپیوترهای کلاسیک دو نوع بسیار متفاوت از رایانه هستند. کامپیوترهای کوانتومی از بیت‌های کوانتومی (کیوبیت‌ها) برای ذخیره اطلاعات استفاده می‌کنند، در حالی که کامپیوترهای کلاسیک از بیت‌ها استفاده می‌کنند. کیوبیت‌ها می‌توانند در حالت برهم‌نهی دو حالت، 0 و 1، به‌طور همزمان باشند، در حالی که بیت‌ها فقط می‌توانند در یک حالت به‌طور همزمان باشند. این تفاوت در نحوه ذخیره اطلاعات امکان محاسباتی را برای کامپیوترهای کوانتومی فراهم می‌کند که برای کامپیوترهای کلاسیک غیرممکن است.

علاوه بر تفاوت در نحوه ذخیره اطلاعات، کامپیوترهای کوانتومی و کلاسیک در نحوه انجام محاسبات نیز متفاوت هستند. کامپیوترهای کوانتومی از مکانیک کوانتوم برای انجام محاسبات استفاده می‌کنند، در حالی که کامپیوترهای کلاسیک از منطق بولی استفاده می‌کنند. این تفاوت در نحوه انجام محاسبات نیز به کامپیوترهای کوانتومی امکان می‌دهد تا برای برخی از وظایف، محاسباتی را بسیار سریع‌تر از کامپیوترهای کلاسیک انجام دهند.

کاربردهای بالقوه کامپیوترهای کوانتومی بسیار گسترده است. آنها می‌توانند برای رمزگشایی روش‌های رمزنگاری فعلی، شبیه‌سازی مولکول‌ها و آموزش مدل‌های یادگیری ماشینی که بسیار دقیق‌تر از مدل‌های فعلی هستند، استفاده شوند. کامپیوترهای کوانتومی هنوز در مراحل اولیه توسعه هستند، اما پتانسیل تغییر جهان را دارند. هنگامی که کامپیوترهای کوانتومی قدرتمندتر شوند، قادر به حل مشکلاتی خواهند بود که برای کامپیوترهای کلاسیک در حال حاضر غیرممکن است.

 برخی از مثال‌های خاص از نحوه استفاده از کامپیوترهای کوانتومی:

\paragraph{رمزنگاری}: کامپیوترهای کوانتومی می‌توانند برای رمزگشایی روش‌های رمزنگاری فعلی استفاده شوند، که تأثیر عمده‌ای بر امنیت آنلاین خواهد داشت.
\paragraph{شبیه‌سازی مواد در علم شیمی}:کامپیوترهای کوانتومی می‌توانند برای شبیه‌سازی مولکول‌ها استفاده شوند، که می‌تواند به دانشمندان در توسعه داروها و مواد جدید کمک کند.
\paragraph{یادگیری ماشینی}: کامپیوترهای کوانتومی می‌توانند برای آموزش مدل‌های یادگیری ماشینی که بسیار دقیق‌تر از مدل‌های فعلی هستند، استفاده شوند.

آینده محاسبات کوانتومی بسیار روشن است. هنگامی که کامپیوترهای کوانتومی قدرتمندتر شوند، قادر به حل مشکلاتی خواهند بود که برای کامپیوترهای کلاسیک در حال حاضر غیرممکن است. این می‌تواند منجر به پیشرفت‌های عمده در بسیاری از زمینه‌های مختلف شود.
\section{شبیه‌سازی کلاسیک در مقابل کامپیوتر کوانتومی}
% computer vs simulator https://www.youtube.com/watch?v=2qcvgXaDdfU
% https://www.youtube.com/watch?v=0RPFWZj7Jm0	

بسیاری از مسائل کوانتومی و بسیاری از الگوریتم‌های کوانتومی قابل شبیه‌سازی روی کامپیوتر‌های کوانتومی می‌باشند. بنابرین یک سوال به‌واقع مهم مطرح می‌شود: \textbf{چرا به یک کامپیوتر کوانتومی نیاز داریم؟}

در هنگام محاسبات کوانتومی،‌ کامپیوتر‌های کلاسیک دارای محدودیت‌هایی هستند. به طور مشابه کامپیوتر‌های کوانتومی نیز دارای معایبی هستند؛ که قابل بحث و بررسی هستند. دراین بخش به این مزایا و معایب هرکدام از کامپیوترها می‌پردازیم و در ادامه به اهداف تعریف شده برای کامپیوترهای کوانتومی می‌پردازیم.


\subsection{تفاوت کامپیوترهای کوانتومی و شبیه‌سازهای کلاسیک}
همانطور که در بخش‌های قبلی گفته شد؛ کامپیوترهای کوانتومی با استفاده از کیوبیت‌ها تعریف می‌شوند. یک کیوبیت به صورت یک ترکیب خطی از حالت $\vert0\rangle$ و $\vert1\rangle$ تعریف می‌شود:

\begin{center}
	$\vert\psi\rangle = \alpha\vert0\rangle + \beta\vert1\rangle$
\end{center}

هریک از حالات داخل رابطه‌ی بالا به صورت یک ماتریس قابل تعریف هستند. به طور مشابه هریک از عملگر های کوانتمی را می‌توان به صورت یک ماتریس تعریف کرد. ماتریس‌هایی مشابه ماتریس پائولی یا در مقیاس‌های بالاتر ماتریس فردکین که برای سه کیوبیت تعریف می‌شود؛ و محاسبات را سریع می‌سازد.



از طرفی دیگر تعداد محاسبات در مدار‌های کلاسیک به تعداد حالات مسأله بستگی دارد. در محاسبات کلاسیک هرچه تعداد حالات بالاتر برود؛ پیچیدگی محاسبات بالاتر‌می‌رود و حتی اغلب با حالت نمایی رشد می‌کنند. این درحالیست که در محاسبات کوانتومی هر یک حالات مختلف مسأله به یک دنیای موازی کوانتومی شیفت داده می‌شود و از این طریق محاسبات به زمان و منابع کمتری نیاز دارد.


مهم‌ترین عامل در سطح پیچیدگی محاسبات کوانتومی همدوسی می‌باشد. 
همدوسی در محاسبات کوانتومی اصلی ترین منبع خطا در این سیستم ها است. کیوبیت ها بسیار به محیط خود هستند و می توانند به راحتی توسط  تعامل با فوتون ها، الکترون ها و سایر ذرات با آن‌ها دچار ناهمدوسی شوند. این می تواند باعث شود که کیوبیت ها خواص کوانتومی خود را مانند برهمنهی کوانتومی و درهمتنیدگی که برای انجام محاسبات کوانتومی ضروری هستند، از دست بدهند.

ناهمدوسی می‌تواند به دلایل مختلفی ایجاد شود:

\paragraph{دما}کیوبیت ها در دماهای بالاتر مستعد دکوراسیون هستند. این به این دلیل است که هرچه دما بالاتر باشد، کیوبیت ها انرژی بیشتری دارند و بیشتر احتمال دارد با محیط خود تعامل داشته باشند.

\paragraph{برخورد}کیوبیت ها همچنین می توانند توسط ارتعاشات دکور شوند. این به این دلیل است که ارتعاشات می توانند باعث حرکت کیوبیت ها شوند، که می تواند حالت های کوانتومی آنها را مختل کند.

\paragraph{تابش الکترومغناطیسی}کیوبیت ها می توانند توسط تابش الکترومغناطیسی، مانند نور و امواج رادیویی، دکور شوند. این به این دلیل است که تابش الکترومغناطیسی می تواند با الکترون های کیوبیت ها تعامل داشته باشد و باعث از دست رفتن خواص کوانتومی آنها شود.

ناهمدوسی یک مانع بزرگ برای توسعه ابررایانه های کوانتومی است. برای ساخت یک رایانه کوانتومی عملی، باید راه هایی برای کاهش ناهمدوسی پیدا کرد. این یک مشکل بزرگ ولی قابل حل است؛ اما تعدادی از مسیرهای تحقیقاتی امیدوار کننده وجود دارد، مانند:

\paragraph{سردسازی کیوبیت‌ها} این می تواند انرژی کیوبیت ها را کاهش دهد و آنها را کمتر مستعد تعامل با محیط خود کند.

\paragraph{استفاده از مواد با همدوسی بالا}برخی از مواد، مانند ابررساناها، زمان های همدوسی بسیار طولانی دارند که آنها را برای استفاده در رایانه های کوانتومی بسیار مناسب می کند.

\paragraph{توسعه‌ی الگوریتم‌های جایگزین برای تصحیح خطای‌‌کوانتومی}
الگوریتم های تصحیح خطا کوانتومی می توانند برای تشخیص و تصحیح خطاهایی که توسط ناهمدوسی ایجاد می شوند استفاده شوند.



\section{شبیه‌سازی کلاسیک}
حال که به اهمیت مسأله‌ی همدوسی در حل محاسبات پی بردیم لازم‌است به نارسایی کامپیوتر‌های کلاسیک در برخورد با این مفهوم بپردازیم:

عکس از چنل یوتیب بذار و کدها رو باهم مقایسه کن.


\newpage
\section{کامپیوترکوانتومی IBM و زبان برنامه‌نویسی QisKit}

\subsection{معرفی QisKit}
یک زبان برنامه‌نویسی کوانتومی است. این زبان دارای مشابهت‌های زیادی با زبان پایتون می‌باشد. دو دلیل عمده برای این شباهت وجود دارد. 1. این زبان براساس زبان پایتون و برخی کتابخانه‌های آن ساخته شده است. 2. جامعه دانشمندان کوانتوم و به‌طور کلی فیزیکدانان از سابق برای انجام شبیه‌سازی‌های خود از زبان پایتون استفاده می‌کنند و پایتون نیز کتابخانه‌هاي بسیار کارآمدی - نظیر نامپای\footnote{Numpy} ، پانداس\footnote{Pandas} ، سایپای\footnote{Scipy} و .. - ارائه کرده است.

\subsection{کدنویسی به زبان QisKit}
برای کدنویسی به زبان \lr{QisKit} می‌توان از محیط \lr{Jupyter notebook} استفاده کرد. پس از نصب \lr{Qiskit} برای فرخواندن این کتابخانه به راحتی می‌توان نوشت:

\begin{latin}
	\begin{verbatim}
		import qiskit
	\end{verbatim}
\end{latin}

هدف از برنامه نویسی کوانتومی پیاده‌سازی مسائل کوانتومی روی کامپیوتر‌کوانتومی‌ واقعی است. بدین منظور ابتدا یک شبیه سازی روی سیستم خود انجام داده و سپس کد خود را به کامپیوتر‌کوانتومی \lr{IBM} ارسال می‌کنیم.بدین منظور می‌بایست یک حساب کاربری در سایت مرتبط به کامپیوتر کوانتومی \lr{IBM} ایجاد کنیم.


\subsection{قواعد نوشتاری و ساختار داده}
هرزبان برنامه نویسی دارای ساختار داده و دستورات نگارشی مخصوص به‌خود است؛ \lr{QisKit} نیز از این قاعده مستثنی نیست. بدین دلیل در ادامه به اختصار به معرفی برخی دستورات این زبان می‌پردازیم. باید خاطر نشان کرد که این متن براساس \lr{QisKit} نسخه‌ی  0.44.0 نوشته‌ شده‌است.

هدف نهایی این برنامه‌ها، پیاده‌سازی یک مدار کوانتومی‌ منتسب به یک مسئله‌ی خاص است. پس نیاز به تعریف و ایجاد کیوبیت داریم. برای تعریف کیوبیت از راه زیر استفاده می‌کنیم:

\begin{latin}
	\begin{verbatim}
		quantum_register = QuantumRegister(n)
	\end{verbatim}
\end{latin}
که در آن n تعداد کیوبیت‌های داخل کوانتوم رجیستر مدنظر برنامه نویس است.


\begin{latin}
	\begin{verbatim}
		classic_register = ClassicalRegister(n)
	\end{verbatim}
\end{latin}
که در آن n تعداد بیت‌های داخل رجیستر کلاسیک مدنظر برنامه نویس است.


درنهایت ‌می‌توان با قرار دادن این دو رجیستر‌ایجاد شده،‌ در یک مدار، یک مدار کوانتومی ایجاد کنیم:


\begin{latin}
	\begin{verbatim}
		circuit = QuantumCircuit(quantum_register, classic_register)
	\end{verbatim}
\end{latin}


پس از ایجاد مدار می‌توان عملگر‌های مدنظر خود نظیر هادامارد و \lr{CNOT} و ... را روی بخش‌های مختلف مدار اثر داد.
از طرفی می‌توان مدار ایجاد شده را به صورت یک تصویر به کاربر نشان داد.


\subsection{پیاده‌سازی گیت‌های کوانتومی}
برای اعمال یک گیت کوانتومی روی اجزای یک مدار کوانتومی،‌ابتدا باید نام مدار کوانتومی را نوشت؛ سپس نام گیت کوانتومی را نوشت؛ سپس نام کیوبیت یا بیت مدنظر را نوشت: 

\begin{latin}
	\begin{verbatim}
		circuit.gate_sing_inQisKit(Qubit_name, Bit_name)
	\end{verbatim}
\end{latin}


در کد بالا به جای \lr{gate\_sing\_inQisKit()} نام هر گیت کوانتومی مدنظر برنامه‌نویس می‌تواند باشد.

\pagebreak
\subsection{تصویر سازی}

برای تصویر کردن وضعیت مدار کوانتومی می‌بایست از راه زیر استفاده کرد:

\begin{latin}
	\begin{verbatim}
		# to run commands and show diagram in a easier way.
		%matplotlib inline 
		
		# draw circuit
		circuit.draw()
		
		# we can change look of diagram.
		circuit.draw(output = "mpl")  # output = "mpl" is defining type of circuit output
		
	\end{verbatim}
\end{latin}



درنهایت خروجی مشابه خروجی های زیر را می‌توان از این زبان برنامه نویسی دریافت کرد:

\begin{center}
	\begin{figure}[ht]
		\centering
		\includegraphics[]{tutorials_circuits_advanced_03_advanced_circuit_visualization_7_0.png}
		\caption{نمونه‌ای از خروجی \lr{Qiskit}}
	\end{figure}
	
\end{center}














































\chapter{الگوریتم‌های‌کوانتومی‌}


چه گونه‌ای از مسائل محاسباتی قابل اجرا با مدارهای کوانتومی می‌باشند؟ 
تفاوت و برتری مدار‌های کوانتومی نسبت به مدار‌های کلاسیک چیست؟
آیا می‌توان یک حوزه‌ی خاص را تعیین کرد؛ به گونه‌ای که عملکرد کامپیوترهای‌‌کوانتومی نسبت به کامپیوتر‌های کلاسیک مزیت داشته باشند؟

در این بخش می‌خواهیم به طور خلاصه این سوالات را پاسخ دهیم و توضیح دهیم چگونه می‌توان از کامپیوتر‌های کوانتومی به شکلی سودمند استفاده کنیم.

\section{موازی سازی کوانتومی}

موازی‌سازی کوانتومی\footnote{Quantum parallelism}، پایه‌واساس بسیاری از الگوریتم‌های کوانتومی است. با گذار یک حالت کوانتومی به حالت برهمنهی کوانتومی، درحین محاسبات کوانتومی یک تابع نظیر \lr{f(x)}، می‌تواند مقادیر مختلف \lr{x} را به طور همزمان بررسی کند. این درحالیست که در محاسبات کلاسیک به دلیل ماهیت بیت‌های اطلاعات، تابع \lr{f(x)} فقط می‌تواند یکی از مقادیر مجاز برای \lr{x} را بررسی کند.


فرض کنید تابع f ،یک تابع تک-کیوبیت، به صورت زیر تعریف شده است:\\
\begin{center}
	$f (x) : \{0, 1\} \rightarrow \{0, 1\}$\\
\end{center}

روش مناسب برای محاسبه این تابع در یک کامپیوتر کوانتومی، با در نظر گرفتن دو کیوبیت که در حالت $\vert x, y\rangle$شروع می شود. با یک توالی مناسب از گیت های منطقی می توان این حالت را به $\vert f(x) \oplus x, y\rangle$ تبدیل کرد که در آن $\oplus$ بیانگر جمع مدوله با پایه‌2 می‌باشد.

\footnote{Modulo 2 is a mathematical operation that returns the remainder of a division by 2. For example, 5 mod 2 is 1, because 5 divided by 2 has a quotient of 2 and a remainder of 1.
	
	The symbol "⊕" is often used to indicate addition modulo 2. So, the expression "5 ⊕ 3" would be evaluated as follows:
	
	5 ⊕ 3 = (5 + 3) mod 2 = 8 mod 2 = 0
	This is because 8 divided by 2 has a quotient of 4 and a remainder of 0.
	
	Modulo 2 is a useful operation in many different areas of mathematics, including cryptography, computer science, and number theory. It is also used in everyday life, for example, when checking whether a number is even or odd.
	
	Here are some other examples of modulo 2:
	
	1 mod 2 = 1
	2 mod 2 = 0
	3 mod 2 = 1
	4 mod 2 = 0
	5 mod 2 = 1}

هریک از دسته‌های کیوبیت، \textbf{رجیستر کوانتومی} نامیده‌می‌شوند. اولین رجیستر، «رجیستر داده» و دومین رجیستر «رجیستر هدف» نامیده می‌شود.\\
ازین پس در این بخش به عامل گذار $\vert x, y \rangle \rightarrow \vert x, y \oplus f(x) \rangle$، عنوان «تابع \lr{$U_{f}$}» را اطلاق خواهیم کرد. لازم به ذکرست که این تبدیل، یک تبدیل یکه به شمار می‌آید.\footnote{اثبات این مطلب از حوصله‌ی بحث خارج است.}


اگر \lr{y = 0} آنگاه مقدار دومین کیوبیت بعد از اعمال تابع \lr{$U_{f}$} برابر با مقدار \lr{f(x)} خواهد بود.

 \begin{center}
 	 \begin{figure}[ht]
 		\centering
 		\includegraphics[width=0.8\textwidth]{Uforacle.png}
 		\caption{مدار کوانتومی برای ارزیابی $f (0)$ و $f (1)$ به طور همزمان. Uf مدار کوانتومی است که ورودی هایی مانند $\vert x, y\rangle$ را به $\vert x, y \rangle \rightarrow \vert x, y \oplus f(x) \rangle$، تصویر می‌کند.}
 	\end{figure}
 \end{center}

در شکل بالا مقادیر ورودی داده شده به تابع \lr{$U_{f}$} در پایه‌های محاسباتی قرار ندارند. رجیستر داده در حالت برهمنهی  قرار دارد. این حالت برهمنهی را می‌توان با اعمال گیت هادامارد بر حالت کوانتومی $\vert 0 \rangle$ ایجاد کرد. پس از ایجاد این حالت، تابع \lr{$U_{f}$} را به حالت جدید اعمال می‌کنیم:
\begin{center}
	$\frac{\vert 0,f(0) \rangle +\vert 1,f(1) \rangle }{\sqrt{2}}$
\end{center}

این یک حالت استثنایی است! جملات مختلف کسر بالا حاوی اطلاعاتی در مورد f (0) وf (1) می‌باشند؛ به‌نحوی که انگار f (x) را برای دو مقدار x به طور همزمان ارزیابی کرده ایم، این ویژگی به "موازی  سازی کوانتومی" موسوم می‌باشد. بر خلاف موازی سازی کلاسیک، که در آن هر یک مدارهای متعددی دارند ساخته شده برای محاسبه f (x) به طور همزمان اجرا می شوند، در اینجا برای ارزیابی تابع برای چندین مقدار x به طور همزمان، یک مدار f (x) (با قابلیت برهمنهی کوانتومی)استفاده می شود.

این فرآیند را می توان به راحتی  با استفاده از یک عمل کلی به نام تبدیل هادامارد، به توابعی با تعداد بیت دلخواه تعمیم داد. این عمل فقط تعداد n گیت هادامارد است که به طور موازی روی n کیوبیت عمل می کنند.

برای مثال در شکل زیر؛ دو کیوبیت در حالت $\vert 0 \rangle$ آماده شده‌‌اند. پس از اعمال گیت‌های هادامارد بر روی این رجیستر به خروجی زیر خواهیم رسید: 

\begin{center}
	$ \Biggl( \frac{\vert 0 \rangle + \vert 1 \rangle}{\sqrt{2}}\Biggl)\Biggl( \frac{\vert 0 \rangle + \vert 1 \rangle}{\sqrt{2}}\Biggl) = \frac{\textbar00\rangle + \textbar01\rangle + \textbar10\rangle + \textbar11\rangle}{\sqrt{2}}$

\end{center}

از نماد $\operatorname{H} \otimes 2$ به عنوان نشانه‌ی عملکرد موازی دو گیت هادامارد استفاده می‌کنیم؛از علامت $\otimes$ به عنوان تانسور یاد می‌کنیم. به طور کلی نتایج اعمال موازی گیت هادامارد روی \lr{n} کیوبیت روی حالت کوانتومی برابرست با:

\begin{center}
	\[\frac{1}{\sqrt{2}} \sum_{x} \vert x \rangle\]
\end{center}


در اینجا، $\sum$ نشان دهنده جمع بر روی همه مقادیر ممکن \lr{x} است، و ما$\operatorname{H} \otimes n$  را برای نشان دادن این عمل می نویسیم.
اعمال تبدیل هادامارد روی یک بهمنهی کوانتومی برابر از همه حالت های محاسباتی تولید می کند؛ و با استفاده از فقط $n$ گیت، یک برهمنهی از $2n$ حالت تولید می کند.
\begin{center}
	\begin{figure}[ht]
		\centering
		\includegraphics[width=0.8\textwidth]{Multyhadamard.png}
		\caption{اعمال تبدیل هادامارد $\operatorname{H} \otimes n$روی دو کیوبیت}
	\end{figure}
\end{center}

تبدیل هادامارد $\operatorname{H} \otimes 2$ روی دو بیت کوانتومی پیاده می‌شود. ارزیابی موازی کوانتومی یک تابع f(x) با ورودی n بیتی x و خروجی 1 بیتی، به روش زیر قابل پیاده‌سازی می‌باشد:

\begin{enumerate}
	\item ابتدا حالت $n + 1$ کیوبیت $\vert0\rangle^{\otimes n} \vert 0\rangle$ را آماده کنید،
	\item سپس تبدیل هادامارد را به $n$ کیوبیت اول و به دنبال آن مدار کوانتومی اعمال کنید.
	\item اعمال تابع $U_{f}$ به کیوبیت‌هایی که در حالت برهمنهی قرار دارند.
\end{enumerate}
درنهایت حالت زیر تولید ‌می‌شود:

1√2n∑x|x〉|f (x)〉

\begin{center}
	$\frac{1}{\sqrt{2^n}} \sum_{x} \vert x \rangle \vert f (x)\rangle$ 
\end{center}


به طور کلی موازی‌سازی کوانتومی امکان ارزیابی همزمان همه مقادیر ممکن تابع f را فراهم می‌کند، حتی اگر ظاهراً فقط یک بار f را ارزیابی کرده باشیم. با این حال، این موازی‌سازی بلافاصله مفید نیست. 


در مثال تک کیوبیتی ما، اندازه‌گیری حالت فقط $\vert 0, f(0)\rangle$ یا $\vert 1, f(1)\rangle$ را می‌دهد! به طور مشابه، در حالت کلی، اندازه‌گیری حالت ∑x |x, f (x)〉 فقط$ f (x)$ را برای یک مقدار x خاص می‌دهد. البته یک کامپیوتر کلاسیک می‌تواند این کار را به راحتی انجام دهد! محاسبات کوانتومی برای مفید بودن به چیزی بیش از موازی‌سازی کوانتومی نیاز دارد؛ به توانایی استخراج اطلاعات مربوط به بیش از یک مقدار f (x) از حالت‌های سوپرپوزیسیون مانند ∑x |x, f (x)〉 نیاز دارد. 
در بخش های بعدی به مثال‌های خواهیم پرداخت که این مسائل را حل کند.



$\sum_{x} \vert x, f(x)\rangle$

\newpage
\subsection{مدل محاسباتی استاندارد}
پیش از بررسی مدل کوئری،‌ مدل ساده و استاندارد محاسباتی را بررسی می‌کنیم. به تصویر زیر دقت کنید:

\begin{figure}[ht]
	\centering
	\includegraphics[width=0.8\textwidth]{standard computation model.png}
	\caption{یک واحد محاسباتی که مقادیری را به عنوان ورودی گرفته، پردازش کرده و سپس مقدار/مقادیر خروجی را ارائه کرده است.}
\end{figure}


در تصویر بالا یک نمود ساده از کامپیوتر‌های امروزی ارائه شده است. در دنیای واقعی مقدار ورودی می‌تواند از هر منبعی‌ تأمین شده باشد. با این وجود هدف ما بررسی منابع تولید ورودی نیست؛‌ بلکه هدف بررسی مقادیر ورودی (به صورت ایزوله) می‌باشد. می‌توان درنظر گرفت که ورودی داده شده و خروجی نهایی،‌هر دو در قالب یک رشته از اعداد باینری، ماتریس و یا هرقالب مدنظر کاربر باشند.

مهم‌ترین نکته درباره‌ی این واحد محاسباتی،‌ \textbf{در دسترس بودن کل مقادیر ورودی برای واحد پردازش} است. به عبارت دیگر\textbf{ واحد پردازش می‌تواند تمامی مقادیر ورودی را دریافت کرده و تشخیص} دهد. 

\section{مدل کوئری}
در مدل کوئری، داده‌های ورودی توسط یک تابع تولید می‌شوند. واحد محاسباتی دسترسی به تابع تولیدورودی دارد و می‌تواند برای دریافت داده‌های جدید،‌از تابع یاد شده،‌درخواست کند.

\begin{figure}[ht]
	\centering
	\includegraphics[width=0.8\textwidth]{Query computation model.png}
	\caption{شکل بالا نمود مدل محاسباتی‌کوئری است. واحد محاسباتی برای دریافت داده‌های جدید نیاز به درخواست از تابع \lr{input} دارد. خطوط قرمز و روبه‌بالا نشان از درخواست واحد محاسباتی و خطوط آبی روبه‌پایین نشان از پاسخ واحد \lr{input}می‌باشد.}
\end{figure}


در این مدل واحد محاسباتی دیگر داده‌ها را در قالب رشته‌ای از اطلاعات دردسترس ندارد؛ بلکه می‌تواند آن‌ها را از بخش \lr{input} دریافت کند. در گاهی از مواقع به سیستم \lr{input}،‌\lr{oracle} یا جعبه‌ی سیاه می‌گویند. تابع \lr{Oracle } یا جعبه‌ی سیاه یک سیستم است که ما به عنوان ناظر به سازوکار داخلی آن و  تمامی اطلاعات آن دسترسی نداریم و فقط می‌توانیم مقادیر مجاز را به آن داده و مقادیر خروجی را دریافت کنیم. 

تابع \lr{oracle} به صورت زیر تعریف می‌شود:
\begin{center}
$$
\left\{
\begin{array}{ll}
f : \sum^n = \sum^m\\
Which : m, n \in \mathbb{N}
\end{array}
\right.
$$
\end{center}

ما در این نظریه کوئری‌ها را می‌شماریم و وضعیت آن‌‌ها را بررسی می‌کنیم.

\section{معرفی و پیاده سازی الگوریتم دوچ}

\subsection{مسئله‌ی دوچ}
الگوریتم \lr{Deutsch} اولین و ساده‌ترین الگوریتم کوانتومی‌ است. این الگوریتم برای اولین بار در سال 1985 در مقاله‌ای مطرح شد؛ که توسط دیوید دوچ\footnote{David Deutsch} نوشته شده‌‌بود. این الگوریتم نقطه‌ی شروعی برای اثبات برتری کامپیوترهای کوانتومی نسبت به کامپوترهای کلاسیک است.

مسئله‌ی \lr{Deutsch} یکی از ساده‌ترین مفاهیم ممکن را مطرح می‌کند. اگر یک تابع به فرم زیر تعریف شود: 
\begin{center}
	$f : \sum \rightarrow \sum$
\end{center}
هدف بررسی ثابت بودن یا متعادل\footnote{Constante or balanse.} بودن تابع \lr{f} است. 
به‌طور کلی، درساده‌ترین حالت، می‌توان چهار وضعیت را برای تابع $f : \sum \rightarrow \sum$ درنظر گرفت:\\
\begin{figure}[ht]
	\centering
	\includegraphics[width=0.8\textwidth]{Constantorbalanse.png}
	\caption{}
\end{figure}\\
در شکل بالا توابع \lr{f 1} , \lr{f 4} توابع ثابت و توابع \lr{f 2} و \lr{f 3} توابع متعادل هستند.
\begin{center}
\begin{tabular}{|c|c|}
	\hline
	\multicolumn{2}{|c|}{مسئله‌ی دوچ} \\
	\hline
	$f : \sum \rightarrow \sum$ & ورودی \\
	\hline
	صفر اگر تابع ثابت بود؛ یک اگر تابع متعادل بود.  & خروجی \\
	\hline
\end{tabular}
\end{center}
در الگوریتم‌های کلاسیک برای حل این مسئله، حداقل دو حالت باید بررسی شود.
\subsection{الگوریتم دوچ}
% quantum computer scinse : read it for yourself
% https://www.youtube.com/watch?v=2wticzHE1vs&t=2014s
% page 32

حال به بررسی الگوریتم دوچ می‌پردازیم. الگوریتمی که مسئله‌ی دوچ را با یک مدار کوانتومی حل می‌کند:\\
\begin{center}
\begin{figure}[ht]
	\centering
	\includegraphics[width=0.8\textwidth]{Deutsch algorithm.png}
	\caption{}
\end{figure}
\end{center}

 
یک تغییر ساده در مدار شکل 1.3 نشان می دهد که چگونه مدارهای کوانتومی می توانند با پیاده سازی الگوریتم Deutsch از مدارهای کلاسیک پیشی بگیرند\footnote{ما در واقع یک نسخه ساده شده و بهبود یافته از الگوریتم اصلی را ارائه می دهیم.}. الگوریتم Deutsch ترکیبی از موازی سازی کوانتومی با خاصیتی از مکانیک کوانتوم به نام تداخل\footnote{how interference is using in deutsch algorithm : Interference is used in the Deutsch algorithm to distinguish between constant and balanced functions. A constant function is a function that always returns the same value, regardless of its input. A balanced function is a function that returns 0 for half of its inputs and 1 for the other half.
	
	The Deutsch algorithm works by first preparing two qubits in a superposition of states. The first qubit is prepared in the state |+⟩, which is the equal superposition of $\vert 0 \rangle$ and $\vert 1 \rangle$. The second qubit is prepared in the state |ψ⟩, which is a superposition of the states $\vert 0 \rangle$ and $\vert 1 \rangle$ with opposite phases.
	
	The two qubits are then passed through a quantum circuit that includes a Hadamard gate and a CNOT gate. The Hadamard gate transforms |+⟩ into the superposition $\vert 0 \rangle$ + $\vert 1 \rangle$, and the CNOT gate copies the state of the first qubit to the second qubit.
	
	After the quantum circuit has been executed, the two qubits are measured. If the first qubit is measured to be $\vert 0 \rangle$, then the function f is constant. This is because the state |ψ⟩ is orthogonal to the state $\vert 0 \rangle$, so the interference between the two states will destructively interfere.
	
	If the first qubit is measured to be $\vert 1 \rangle$, then the function f is balanced. This is because the state |ψ⟩ is parallel to the state $\vert 1 \rangle$, so the interference between the two states will constructively interfere.
	
	The Deutsch algorithm is a simple example of how quantum interference can be used to solve a problem that is difficult to solve classically. In this case, the Deutsch algorithm can distinguish between constant and balanced functions in a single step, while a classical computer would need to take an exponential number of steps.} است.
مشابه قبل،‌ابتدا از گیت هادامارد برای آماده سازی اولین کیوبیت به عنوان superposition ($vert 0 \rangle$ + |1〉)/√2 استفاده کنیم، اما اکنون کیوبیت دوم y را با اعمال یک گیت هادامارد به حالت |1> به عنوان superposition ($vert 0 \rangle$ - |1〉)/√2 آماده کنیم. به‌شکل زیر دقت کنید:



\begin{figure}[ht]
	\centering
	\includegraphics[width=0.8\textwidth]{Quantum circuit implementing Deutsch’s algorithm.png}
	\caption{پیاده‌سازی مدارکوانتومی‌ الگوریتم دوچ}
\end{figure}

\newpage

حالت ورودی:\\
\begin{center}
	$\vert\psi_{0}\rangle = \vert01\rangle$
\end{center}

سیستم دو کیوبیتی تشکیل شده،‌ پس از اعمال اثر دو گیت هادامارد می‌دهد:\\

\begin{center}
$\left|\psi_1\right\rangle=\left[\frac{|0\rangle+|1\rangle}{\sqrt{2}}\right]\left[\frac{|0\rangle-|1\rangle}{\sqrt{2}}\right]$
\end{center}
با کمی تأمل می‌توان دریافت که اگر Uf را به حالت |x〉($vert 0 \rangle$ - |1〉)/√2 اعمال کنیم، سپس به حالت (−1)f (x)|x〉($vert 0 \rangle$ - |1〉)/√2 می رسیم. بنابرین اعمال Uf به |ψ1〉 ما را با یکی از دو امکان زیر مواجه می کند:
\begin{center}
	$\left|\psi_2\right\rangle= \begin{cases} \pm\left[\frac{|0\rangle+|1\rangle}{\sqrt{2}}\right]\left[\frac{|0\rangle-|1\rangle}{\sqrt{2}}\right] & \text { if } f(0)=f(1) \\ \pm\left[\frac{|0\rangle-|1\rangle}{\sqrt{2}}\right]\left[\frac{|0\rangle-|1\rangle}{\sqrt{2}}\right] & \text { if } f(0) \neq f(1) \end{cases}$
\end{center}
با اعمال آخرین گیت هادامارد روی کیوبیت اول به حالت زیر خواهیم رسید:
\begin{center}
	$\left|\psi_3\right\rangle= \begin{cases} \pm|0\rangle\left[\frac{|0\rangle-|1\rangle}{\sqrt{2}}\right] & \text { if } f(0)=f(1) \\ \pm|1\rangle\left[\frac{|0\rangle-|1\rangle}{\sqrt{2}}\right] & \text { if } f(0) \neq f(1) \end{cases}$
\end{center}
با درنظر گرفتن شرایط زیر می‌توان $\vert\psi_{3}\rangle$ را به شکل زیر بازنویسی کرد:

$$
\begin{aligned}
	\left\{
	\begin{aligned}
		f(0) &= f(1) 
		\implies \qquad f(0) \oplus f(1) &= 0\\
		f(0) &\neq f(1) 
		\implies \qquad f(0) \oplus f(1) &= 1
	\end{aligned}
	\right.
\end{aligned}$$

از این رو:

\begin{center}
	$\left|\psi_3\right\rangle= \pm|f(0) \oplus f(1)\rangle\left[\frac{|0\rangle-|1\rangle}{\sqrt{2}}\right]$
\end{center}
بنابراین با اندازه گیری کیوبیت اول می توانیم $f(0) \oplus f(1)$ را تعیین کنیم. واقعاً جالب است! مدار کوانتومی به ما توانایی تعیین یک ویژگی جهانی از $f(x)$، یعنی$ f(0)\oplus f(1)$ را داده است، با استفاده از تنها یک ارزیابی از $f(x)$!این سریعتر از آن چیزی است که با یک دستگاه کلاسیک امکان‌پذیر است،یک دستگاه کلاسیک حداقل به دو ارزیابی نیاز دارد.
این مثال تفاوت بین موازی‌سازی کوانتومی و الگوریتم‌های تصادفی کلاسیک را برجسته می‌کند. به سادگی، ممکن است تصور شود که حالت $\vert0\rangle \vert f(0) \rangle + \vert1\rangle \vert f(1) \rangle$ مطابقت نزدیکی با یک رایانه کلاسیک تصادفی دارد که هرکدام از حالات $ f (0)$ یا$ f (1) $ با احتمال $1/2$ اندازه‌گیری می‌کند.
تفاوت این است که در یک رایانه کلاسیک این دو گزینه همیشه یکدیگر را حذف می‌کنند. در یک رایانه کوانتومی، امکان دارد که دو گزینه با یکدیگر تداخل داشته باشند تا برخی از خواص کلی تابع $f$ را با استفاده از چیزی شبیه به گیت هادامارد برای بازترکیب گزینه‌های مختلف، مانند آنچه در الگوریتم دوچ انجام شد، به دست آورند.
اساس طراحی بسیاری از الگوریتم‌های کوانتومی این است که یک انتخاب هوشمندانه از تابع و تبدیل نهایی اجازه می‌دهد تا اطلاعات جهانی مفیدی در مورد تابع تعیین شود - اطلاعاتی که نمی‌توان به سرعت در یک رایانه کلاسیک به دست آورد.


	
\section{ساخت یک اوراکل کوانتومی}
بیایید راه های مختلفی را ببینیم که می توانیم یک اوراکل کوانتومی ایجاد کنیم.

برای یک تابع ثابت، ساده است:

اگر f(x) = 0، گیت را به کیوبیت در ثبات 2 اعمال کنید.
اگر f(x) = 1، گیت را به کیوبیت در ثبات 2 اعمال کنید.

برای عملکرد متعادل، مدارهای مختلفی وجود دارد که می توانیم ایجاد کنیم. یکی از راه‌هایی که می‌توانیم متوازن بودن مدار خود را تضمین کنیم، انجام یک CNOT برای هر کیوبیت در ثبات 1، با کیوبیت موجود در ثبات 2 به عنوان هدف است. مثلا:

\begin{center}
	\begin{figure}[ht]
		\centering
		\includegraphics[width=0.8\textwidth]{oraclecon.png}
		\caption{}
	\end{figure}
\end{center}




در تصویر بالا، سه کیوبیت بالا، رجیستر ورودی را تشکیل می دهند و کیوبیت پایین، ثبات خروجی است. در جدول زیر می‌توانیم ببینیم کدام حالت‌های ورودی کدام خروجی را می‌دهند:
\begin{table}[h]
	\centering
	\begin{tabular}{|c|c|}
		\hline
		حالاتی ورودی که به خروجی صفر منجر می‌شوند &  حالاتی ورودی که به خروجی صفر منجر می‌شوند \\ \hline
		000 & 001 \\ \hline
		011 & 100 \\ \hline
		101 & 010 \\ \hline
		110 & 111 \\ \hline
	\end{tabular}
	\caption{This is a simple table.}
\end{table}
ما می توانیم نتایج را تغییر دهیم و در عین حال تعادل آنها را با قرار دادن کنترل های انتخاب شده در X-Gates حفظ کنیم. برای مثال، مدار و جدول نتایج آن را در زیر ببینید:
\begin{center}
	\begin{figure}[ht]
		\centering
		\includegraphics[width=0.8\textwidth]{oraclebalance.png}
		\caption{}
	\end{figure}
\end{center}


\begin{table}[h]
	\centering
	\begin{tabular}{|c|c|}
		\hline
		حالاتی ورودی که به خروجی صفر منجر می‌شوند &  حالاتی ورودی که به خروجی صفر منجر می‌شوند \\ \hline
		001 & 000 \\ \hline
		010 & 011 \\ \hline
		100 & 101 \\ \hline
		111 & 110 \\ \hline
	\end{tabular}
	\caption{This is a simple table.}
\end{table}



% implimentation : https://learn.qiskit.org/course/ch-algorithms/deutsch-jozsa-algorithm

\chapter{شبیه‌سازی پدیده‌های کوانتومی}
در این بخش قصد داریم به بررسی پروتکل‌های ابتدایی در نظریه‌ی اطلاعات کوانتومی بپردازیم. تمامی این پروتکل‌ها به تعداد کمی کیوبیت نیاز دارند؛ و در آزمایشگاه به صورت تجربی پیاده‌سازی شده‌اند. 
\section{Bell states}
% https://www.youtube.com/watch?v=I0jH1_H3x1o
% pls search about bell eq
در اغلب موارد، سیستم از دو کیوبیت درهمتنیده تشکیل شده‌است. تابع حالت این سیستم‌ها به شکل زیر تعریف می‌شوند:
\begin{center}
	$\vert \psi_{00} \rangle = \frac{1}{\sqrt{2}}(|00\rangle + |11\rangle)$
\end{center}
برای آماده‌سازی این حالت کوانتومی‌، در حالت \lr{$\vert0\rangle$} داریم. با اعمال یک گیت هدامارد روی یکی از کیوبیت‌ها و سپس با کنترل‌ آن با گیت \lr{CNOT}، (به نحوی که کیوبیت دوم هدف قرارگیرد.) می‌توان به یک حالت درهمتنیده رسید. می‌توان این مراحل را به شکل زیر شرح داد:
\begin{center}
	$\vert\psi_{00}\rangle = C_{10}H_{1}\vert00\rangle$
\end{center}

%می‌توان رابطه‌ی بالا را به هریک از حالات $\vert00\rangle, \vert 01\rangl, \vert 10\rangl, \vert 11\rangl$ تعمیم داد:%
\[
\vert\psi_{xy}\rangle = C_{10}H_{1}\vert xy \rangle
\]

از آنجایی که چهار حالت $\vert xy\rangle$ یک مجموعه متعامد هستند و گیت‌های هادامارد و cNOT یکه هستند، چهار حالت درهم تنیده $\vert \psi_{xy}\rangle$ نیز یک مجموعه متعامد هستند، که به نام پایه Bell نامگذاری شده‌اند.  می‌توان رابطه‌ی بالا را یک حالت کلی تعمیم داد:

% read about bell state equation
\begin{center}
$\vert \psi_{xy}\rangle = C10H1X^{x}_{1} X^{y}_{0} \vert00\rangle$\\
\end{center}

\begin{figure}
	\centering
	\begin{subfigure}[b]{}
		\centering
		\includegraphics[width=0.2\textwidth]{psi00.png}
		\caption{$y=x$}
		\label{fig:y equals x}
	\end{subfigure}
	\hfill
	\begin{subfigure}[b]{}
		\centering
		\includegraphics[width=0.2\textwidth]{psixy.png}
		\caption{$y=3\sin x$}
		\label{fig:three sin x}
	\end{subfigure}
	\caption{(a) A circuit that creates the entangled state
		|ψ00〉 = 1√2
		(|00〉 + |11〉) from the unentangled computational-basis
		state |00〉. (b) A circuit that creates the four orthonormal entangled
		Bell states |ψxy〉 from the unentangled computational-basis state |xy〉}
	\label{fig:three graphs}
\end{figure}


درنظر داشته باشید که حالات بل قابل تبدیل به یکدیگرند؛ با کمک رابطه‌ی تعمیم یافته ‌می‌توان از هر حالت بل به حالت $\vert00 \rangle$ رسید.
توضیحات درباره‌ی تبدیلات بنویس

این روابط به شکل‌های متعدد قابل بیان است که از حوصله‌ی بحث خارج است.
\newpage






\section{entanglement}

Entanglement: The Mysterious Connection in Quantum Physics

In the realm of quantum physics, one of the most baffling phenomena that continues to capture the imagination of scientists and philosophers alike is "entanglement." This concept challenges our conventional understanding of reality and has given rise to countless debates, experiments, and philosophical inquiries.

At its core, entanglement refers to the extraordinary connection that can exist between two or more particles, regardless of the distance separating them. When particles become entangled, their properties, such as spin, momentum, and polarization, become linked in such a way that the state of one particle instantaneously influences the state of the other, no matter the physical space between them. This instantaneous connection seems to defy classical notions of cause and effect and has led Albert Einstein to dub it "spooky action at a distance."

One of the most famous thought experiments illustrating entanglement is the Einstein-Podolsky-Rosen (EPR) paradox. In this scenario, two entangled particles, often referred to as "Alice" and "Bob," are separated and sent to different locations. When a measurement is performed on one particle, its state is instantaneously determined, causing the other particle's state to be known as well, even if they are light-years apart. This apparent violation of the speed of light's limit has challenged our understanding of how information can be transmitted.

Entanglement is not just a theoretical concept; it has been experimentally observed and verified through a variety of experiments. One of the most groundbreaking experiments demonstrating entanglement is the Bell test, which tests the violation of Bell inequalities. When the measurements on entangled particles violate these inequalities, it strongly suggests that their behavior cannot be explained by classical physics and instead points to the reality of quantum entanglement.

The implications of entanglement are profound and extend beyond the confines of physics laboratories. Researchers are actively exploring the potential applications of entanglement in fields like quantum computing and quantum cryptography. Entanglement's ability to enable qubits (quantum bits) to exist in multiple states simultaneously has the potential to revolutionize computation and cryptography, leading to solutions that were previously considered impossible.

However, entanglement also presents us with philosophical and conceptual challenges. The nature of the instantaneous connection between entangled particles challenges our understanding of space, time, and causality. It has sparked debates about the true nature of reality and the role of human observation in collapsing the quantum wave function.

In conclusion, entanglement stands as one of the most intriguing and perplexing phenomena in the realm of quantum physics. Its mysterious connection between particles defies classical intuitions and continues to push the boundaries of our understanding of the universe. As scientists delve deeper into the intricacies of entanglement and its potential applications, it's clear that this phenomenon will continue to captivate and inspire generations to come.


\newpage



\section{رمزگذاری متراکم کوانتومی}
% from quantum computation l chuang

رمزگذاری متراکم کوانتومی\footnote{\lr{Quantum super dense coding}} یک کاربرد ساده اما شگفت‌انگیز از مفاهیم ابتدایی مکانیک کوانتومی است. این کاربرد، همه ایده‌های اساسی و ابتدایی مکانیک کوانتومی را به روشی ملموس و غیرقابل‌توضیح ترکیب می‌کند، بنابراین مثالی ایده‌آل از اهداف و وظایف پردازش اطلاعات است که می‌توان با استفاده از مکانیک کوانتومی انجام داد.

رمزگذاری متراکم شامل دو طرفین است که به طور معمول به عنوان «آلیس» و «باب» شناخته می‌شوند، که از هم فاصله زیادی دارند. هدف آنها انتقال برخی اطلاعات کلاسیک از آلیس به باب است. فرض کنید آلیس قصد دارد دوبیت داده‌ی کلاسیک را برای باب ارسال کند، اما فقط مجاز است یک کیوبیت به باب ارسال کند. آیا می‌تواند به هدف خود برسد؟

رمزگذاری متراکم به ما می‌گوید که پاسخ این سؤال بله است. فرض کنید آلیس و باب در ابتدا یک جفت کیوبیت در حالت درهم‌تنیده زیر به اشتراک می‌گذارند:

|ψ〉 = |00〉 + |11〉
√2 . (2.133)\\
=========

بیانات چی بیتی
آلیس می‌تواند با استفاده از این جفت کیوبیت‌های درهم‌تنیده، دو بیت کلاسیک را به باب منتقل کند. او این کار را با اعمال یک تبدیل واحد به کیوبیت خود انجام می‌دهد، بسته به اینکه می‌خواهد کدام دو بیت کلاسیک را به باب ارسال کند. به عنوان مثال، اگر آلیس می‌خواهد بیت‌های 00 را به باب ارسال کند، او تبدیل 1 را به کیوبیت خود اعمال می‌کند.

این تبدیل واحد باعث می‌شود که کیوبیت آلیس و کیوبیت باب در یکی از چهار حالت بل اورتوگونال قرار گیرند. اگر آلیس کیوبیت خود را در حالت |00〉 قرار دهد، کیوبیت باب به طور خودکار به حالت |00〉 تبدیل می‌شود. این بدان معناست که آلیس دو بیت اطلاعات (0 و 0) را به باب منتقل کرده است.

باب سپس کیوبیت خود را اندازه‌گیری می‌کند و مقدار 0 یا 1 را به دست می‌آورد. بسته به مقداری که باب اندازه‌گیری می‌کند، او می‌تواند دو بیت کلاسیک که آلیس به او ارسال کرده است را بازیابی کند.

رمزگذاری متراکم یک پروتکل کوانتومی بسیار کارآمد است که می‌تواند دو بیت کلاسیک را با ارسال یک کیوبیت منتقل کند. این پروتکل می‌تواند در کاربردهای مختلفی مانند رمزنگاری کوانتومی و ارتباطات کوانتومی استفاده شود.\\
==========\\

همانطور که در شکل 2.3 قابل ملاحظه است؛ آلیس و باب، هرکدام یک کیوبیت در اختیار دارند. توجه داشته باشید که $\Psi$ یک حالت ثابت است؛ نیازی نیست که آلیس برای آماده‌سازی این حالت، کیوبیتی را به باب ارسال کند. در عوض، ممکن است یک طرف ثالث قبلاً حالت درهم‌تنیده را آماده کند، یکی از کیوبیت‌ها را به آلیس و دیگری را به باب ارسال کند.

با ارسال تک کیوبیت آلیس  به باب، معلوم می‌شود که آلیس می‌تواند دو بیت اطلاعات کلاسیک را به باب منتقل کند. در اینجا روشی که او استفاده می‌کند؛ آورده شده است. اگر او بخواهد رشته‌بیت :

\begin{itemize}
	\item  $"00"$ را ارسال کند $\impliedby$ هیچ کاری روی کیوبیت خود انجام ندهد.
	\item $"01"$ را ارسال کند $\impliedby$ تبدیل دگرگونی فاز\footnote{phase flip} را روی کیوبیت خود اثر می‌دهد.
	\item  $"10"$ را ارسال کند $\impliedby$ گیت کوانتومی NOT، X، را به کیوبیت خود اعمال می‌کند.
	\item  $"11"$ را ارسال کند $\impliedby$ تبدیل iY را به کیوبیت خود اعمال می‌کند.
\end{itemize}
چهار حالت حاصل به راحتی قابل مشاهده هستند:\\
00 : |ψ〉 → |00〉 + |11〉
√2 (2.134)
01 : |ψ〉 → |00〉 − |11〉
√2 (2.135)
10 : |ψ〉 → |10〉 + |01〉
√2 (2.136)
11 : |ψ〉 → |01〉 − |10〉
√2 . (2.137)\\


همانطور که در بخش 1.3.6 اشاره کردیم، این چهار حالت به عنوان پایه بل، حالت‌های بل، یا جفت‌های EPR شناخته می‌شوند، به احترام چندین تن از پیشگامان که اولین بار از نوآوری درهم‌تنیدگی قدردانی کردند. توجه داشته باشید که حالت‌های بل یک پایه متعامد هستند، و بنابراین می‌توانند با اندازه‌گیری کوانتومی مناسب از یکدیگر متمایز شوند. اگر آلیس کیوبیت خود را به باب بفرستد، و باب هر دو کیوبیت را در اختیار داشته باشد، سپس با انجام اندازه‌گیری در پایه بل، باب می‌تواند تعیین کند که کدام یک از چهار رشته بیت ممکن را آلیس ارسال کرده است.

به طور خلاصه می‌توان گفت:
 آلیس، با تعامل و اثرگذاری تنها روی یک کیوبیت، قادر به انتقال دو بیت اطلاعات به باب است. البته دو کیوبیت در پروتکل دخیل هستند، اما آلیس هرگز نیازی به تعامل با کیوبیت دوم ندارد. از نظر کلاسیک، وظیفه‌ای که آلیس انجام می‌دهد، اگر فقط یک بیت کلاسیک ارسال می‌کرد، غیرممکن بود.

علاوه بر این، پروتکل رمزگذاری متراکم، تا حدی در آزمایشگاه تأیید شده است.  یک نکته کلیدی را می‌توان در این مثال زیبا مشاهده کرد: اطلاعات فیزیکی است، و نظریه‌های فیزیکی شگفت‌انگیز مانند مکانیک کوانتوم ممکن است توانایی‌های پردازش اطلاعات شگفت‌انگیزی را پیش‌بینی کنند.



بیانات چی پی تی:
پروتکل سوپردنس کدینگ یک پروتکل کوانتومی است که آلیس می‌تواند از آن برای انتقال دو بیت کلاسیک به باب با ارسال یک کیوبیت استفاده کند. پروتکل به شرح زیر است:

آلیس و باب یک جفت کیوبیت درهم‌تنیده را با هم به اشتراک می‌گذارند.
آلیس می‌خواهد دو بیت کلاسیک، 00 یا 01 را به باب ارسال کند.
آلیس یک عمل واحد بر روی کیوبیت خود اعمال می‌کند که بسته به بیت‌هایی که می‌خواهد به باب ارسال کند متفاوت است.
آلیس کیوبیت خود را به باب می‌فرستد.
باب کیوبیت را اندازه‌گیری می‌کند و دو بیت کلاسیکی که آلیس به او ارسال کرده است را دریافت می‌کند.
اگر آلیس یک عمل واحد بر روی کیوبیت خود اعمال نکند، باب فقط می‌تواند یک بیت کلاسیک را دریافت کند. با این حال، اگر آلیس عمل واحد مناسب را اعمال کند، می‌تواند دو بیت کلاسیک را با ارسال یک کیوبیت انتقال دهد.

پروتکل سوپردنس کدینگ یک مثال عالی از قدرت مکانیک کوانتوم در انتقال اطلاعات است. این پروتکل نشان می‌دهد که می‌توان با استفاده از قوانین مکانیک کوانتوم، اطلاعات را به روش‌های غیرممکن در فیزیک کلاسیک انتقال داد.
\newpage
\section{دوربری}
isac book
فرض کنید آلیس یک کیوبیت در حالت زیر دارد:
\begin{center}
$\vert \psi \rangle = \alpha\vert 0 \rangle + \beta\vert 1 \rangle $\\	
\end{center}

	کارول ممکن است با اعمال یک عملگر یکه‌ به یک کیوبیت در حالت استاندارد، کیوبیت را از حالت $\vert0\rangle$ به حالت $\vert\psi\rangle$  تبدیل کرده باشد. کارول بدون اعلام نوع عملگر یکه به آلیس،‌ کیوبیت را برای او ارسال می‌کند. 
	حال آلیس می‌خواهد بدون دسترسی داشتن به کیوبیت باب، تغییراتی را در کیوبیت او ایجاد کند؛ این تنها درصورتی ممکن است که کیوبیت باب و آلیس درهمتنیده باشند. هرچند آلیس و باب می‌توانند از طریق راه‌های کلاسیک(نظیر تلفنی صحبت کردن و ...) با یکدیگر ارتباط برقرار کنند؛ ولی نمی‌توانند دسترسی مستقیم به کیوبیت یکدیگر داشته باشند. 
	
	کیوبیت باب را می‌توان به حالت زیر تعریف کرد:
$\vert\phi\rangle = 1/\sqrt{2}(\vert0\rangle \vert0\rangle + \vert1\rangle\vert1\rangle)$ 
	
he no-cloning theorem prohibits duplicating the unknown state
of Alice’s first Qbit, either far away from her or nearby. But it turns
out to be possible for Alice and Bob to cooperate over the telephone
in assigning the state |ψ〉 to Bob’s member of the entangled pair. The
no-cloning theorem is not violated because in doing so Alice obliterates
all traces of the state |ψ〉 from either of her own Qbits. The process –
called teleporting the state from Alice to Bob – also eliminates the en-
tanglement Alice and Bob formerly shared. For each shared entangled
pair, they can teleport just a single 1-Qbit state. The term “teleporta-
tion” emphasizes that the state assignment acquired by Bob’s Qbit no
longer applies to Alice’s; it has been transported from her Qbit to his.
Here is how teleportation works. Alice’s first Qbit and the entangled
pair she shares with Bob are characterized by the 3-Qbit state
%|ψ〉a |〉ab = (α$vert 0 \rangle$a + β|1〉a
) 1√2
($vert 0 \rangle$a $vert 0 \rangle$b + |1〉a |1〉b
), (6.21)
where I have given the state symbols for the Qbits in Alice’s and Bob’s
possession subscripts a and b . To teleport the unknown state of her
Qbit to Bob’s member of the entangled pair, Alice first applies a cNOT
gate, using her first Qbit in the state |ψ〉 as the control and her member
of the shared entangled pair as the target. This produces the 3-Qbit
state
α$vert 0 \rangle$a 1√2
($vert 0 \rangle$a $vert 0 \rangle$b + |1〉a |1〉b
) + β|1〉a 1√2
(|1〉a $vert 0 \rangle$b + $vert 0 \rangle$a |1〉b
).
(6.22)
Next she applies a Hadamard transformation H to her first Qbit, giving
all three Qbits the state
α 1√2
($vert 0 \rangle$a + |1〉a
) 1√2
($vert 0 \rangle$a $vert 0 \rangle$b + |1〉a |1〉b
)
+ β 1√2
($vert 0 \rangle$a − |1〉a
) 1√2
(|1〉a $vert 0 \rangle$b + $vert 0 \rangle$a |1〉b
)
= 1
2 $vert 0 \rangle$a $vert 0 \rangle$a
(α$vert 0 \rangle$b + β|1〉b
) + 1
2 |1〉a $vert 0 \rangle$a
(α$vert 0 \rangle$b − β|1〉b
)
+ 1
2 $vert 0 \rangle$a |1〉a
(α|1〉b + β$vert 0 \rangle$b
) + 1
2 |1〉a |1〉a
(α|1〉b − β$vert 0 \rangle$b
). (6.23)
Now Alice measures both Qbits in her possession. (As remarked
in Section 6.4, such an application of cNOT and Hadamard gates,
immediately followed by measurement gates, is called “measuring in
the Bell basis.”) If the result is 00, Bob’s Qbit will indeed acquire the
state |ψ〉 originally possessed by Alice’s first Qbit (whose state would
then be reduced to $vert 0 \rangle$). But if the result of Alice’s measurement is 10,
01, or 11 then the state of Bob’s Qbit becomes
α$vert 0 \rangle$b − β|1〉b , α|1〉b + β$vert 0 \rangle$b , or α|1〉b − β$vert 0 \rangle$b . (6.24)
In each of these three cases there is a unitary transformation that re-
stores the state of Bob’s Qbit to Alice’s original state |ψ〉. In the first
case we can apply Z (which leaves $vert 0 \rangle$ alone but changes the sign of |1〉),
in the second case, X (which interchanges $vert 0 \rangle$ and |1〉), and in the third
case, ZX.
6 . 5 TE L E P O R T A T I O N 151
So all Alice need do to transfer the state of her Qbit to Bob’s member
of their entangled pair is to telephone Bob and report to him the results
of her two measurements. He then knows whether the state has already
been transferred (if Alice’s result is 00) or what unitary transformation
he must apply to his member of the entangled pair in order to com-
plete the transfer (if Alice’s result is one of the other three.) Note the
resemblance to quantum error correction: by making a measurement
Alice acquires the information needed for Bob to reconstruct a partic-
ular quantum state, without anybody acquiring any information about
what the state actually is.
This appears to be remarkable. A general state of a Qbit is described
by two complex numbers α and β that take on a continuum of values,
constrained only by the requirement that |α|2 + |β|2 = 1. Yet, with the
aid of a standard entangled pair, whose state does not depend on α and
β, Alice is able to provide Bob with a Qbit described by the unknown
state, at the price of only two bits of classical information (giving the
results of her two measurements) and the loss of the entanglement of their pair.






But of course the teleportation process does not communicate to Bob the information that can be encoded in α and β. Bob is no more able to learn the values of α and β from manipulating his Qbit, now assigned the state |ψ〉, than Alice was able to do when it was her Qbit that was assigned the same state |ψ〉. On the other hand Alice’s state could be produced at a crucial stage of an elaborate quantum computation, and its transfer to Bob could enable him to continue with the computation on his own far-away quantum computer, s  one can achieve a nontrivial objective by such teleportations. Like dense coding, teleportation can also be constructed by manipulating an elementary classical circuit diagram, without going through any of the analysis in (6.21)–(6.24). Figure 6.5(a) shows a circuit that exchanges the state |ψ〉 = |x〉 of Alice’s Cbit with the state $vert 0 \rangle$ of Bob’s Cbit, regardless of whether x = 0 or 1. The transfer is achieved by direct physical coupling between the two Cbits. As a linear quantum circuit it continues to perform this exchange for arbitrary superpositions, |ψ〉 = α$vert 0 \rangle$ + β|1〉. The entire teleportation protocol can be constructed by appropriately expanding the two gates in Figure 6.5(a), with the aid of an ancillary Qbit. The aim of the expansion is to eliminate the direct interaction between Alice’s and Bob’s Qbits through the two cNOT gates in Figure 6.5(a), in favor of the telephoned message from Alice to Bob, and the interaction necessary to produce their shared pair of entangled Qbits (which can take place well before Alice has even acquired her Qbit in the state |ψ〉).





%In Figure 6.5(b) an ancillary Qbit, not acted upon throughout the process, is introduced in the state |φ〉 = H$vert 0 \rangle$ = 1√2 ($vert 0 \rangle$ + |1〉). (6.25) In Figure 6.5(c) the identities X = HZH and 1 = HH have been used to rewrite the cNOT gate on the right of Figure 6.5(b), and an additional cNOT gate has been added on the left, which acts as the identity, since X acts as the identity on the state H$vert 0 \rangle$. Figure 6.5(d) follows from Figure 6.5(c) because the action of CZ is independent of which Qbit is the control and which the target, and because the two cNOT gates on the left of Figure 6.5(c) have exactly the same action as the three cNOT gates on the left of Figure 6.5(d): acting on the computational basis, both sets of gates apply X on both of the bottom two wires if the state of the top wire is |1〉 and act as the identity if the state of the top wire is $vert 0 \rangle$. Figure 6.5(e) follows from Figure 6.5(d) if we write the |φ〉 on the left of Figure 6.5(d) as H$vert 0 \rangle$ and explicitly write the $vert 0 \rangle$ on the right of Figure 6.5(d) as H|φ〉. But Figure 6.5(e) is an automated version of teleportation. To relate it to ordinary teleportation, introduce mea- surements of the upper two Qbits after the circuit of Figure 6.5(e) has acted, as in Figure 6.5(f). Their effect is to collapse the states of each of the two upper wires randomly and independently to $vert 0 \rangle$ or |1〉. But as noted in Section 3.6, measurement of a control Qbit commutes with any operation controlled by that Qbit, so the measurement gates can be moved to the positions they occupy in Figure 6.5(g). Figure 6.5(g) is precisely the teleportation protocol. The two gates on the left transform the two lower Qbits into the entangled state (6.20). The subsequent applications to the top two Qbits of cNOT followed by H followed by two measurement gates are precisely Alice’s “measurement in the Bell basis.” Since Alice knows the outcomes of the measurements, she knows whether the subsequent cNOT and CZ gates will or will not act, and she can replace these physical couplings by a phone call to Bob telling him whether or not to apply X and/or Z directly to his own Qbit. Figure 6.6 demonstrates that entanglement can also be teleported. The figure reproduces parts (b), (e), and (g) of Figure 6.5 with three changes. (1) A bar representing n Qbits in the n-Qbit state |〉i has been added above each part of the figure. No operations act on these additional Qbits. (2) The state to be teleported has been given a subscript i so it is now one of several possible states |ψi 〉. (3) Because of the linearity of the unitary gates we may sum over the index i . The effect of the circuit is to transfer participation in the entangled state ∑ i |i 〉|ψi 〉 from the third wire from the bottom to the bottom wire. So even if Alice’s Qbit has no state of its own but is entangled with other Qbits, Alice can use the same protocol to teleport its role in the entangled state over to Bob’s Qbit. The result is that Bob’s Qbit becomes entangled in exactly the same way Alice’s was, and Alice’s Qbit becomes entirely unentangled.
1
\newpage
how to determine a oracle\\
	
	
	
	
% https://learn.qiskit.org/course/ch-algorithms/quantum-circuits
% ket + and ket - must be deffined.
	
	
\end{document}